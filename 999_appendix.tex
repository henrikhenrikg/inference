\section{Appendix}
\label{sec:appendix}

We heavily rely on Hugging Face's Transformers library \cite{wolf-etal-2020-transformers} for all experiments involving the PLMs.
We used Weights \& Biases for tracking and logging the experiments \cite{wandb}.
Finally, we used sklearn \cite{scikit-learn} for other ML-related experiments.

% \begin{table*}[t]
% \small
    \centering
\resizebox{1\textwidth}{!}{%
\begin{tabular}{lrrrrrrr}
\toprule
                                index &  n\_patterns &  n\_edges &  syntactic &  lexical &  both &  uni &   bi \\
\midrule
                        field of work &          26 &      650 &         16 &       98 &   536 &    0 &  650 \\
                           occupation &          16 &      240 &          0 &       40 &   200 &    0 &  240 \\
                          named after &          21 &      381 &         44 &       35 &   302 &    0 &  381 \\
                       place of birth &          14 &      182 &          4 &       20 &   158 &    0 &  182 \\
                       place of death &          12 &      110 &          0 &       31 &    79 &   20 &   90 \\
                        position held &          28 &       57 &          0 &        7 &    50 &   51 &    6 \\
                     original network &          37 &     1050 &         20 &       77 &   953 &  258 &  792 \\
                   shares border with &          16 &      140 &         20 &       19 &   101 &   68 &   72 \\
 position played on team / speciality &           9 &       72 &         28 &        8 &    36 &    0 &   72 \\
             language of work or name &          18 &      228 &         12 &       14 &   202 &   12 &  216 \\
                    official language &          12 &      132 &         18 &        2 &   112 &    0 &  132 \\
 original language of film or TV show &          21 &      285 &         12 &       62 &   211 &  117 &  168 \\
                         manufacturer &          25 &      600 &         16 &       90 &   494 &    0 &  600 \\
                            developer &           0 &        0 &          0 &        0 &     0 &    0 &    0 \\
                  diplomatic relation &          19 &      342 &         70 &       50 &   222 &    0 &  342 \\
                             owned by &          23 &      141 &          0 &       25 &   116 &   87 &   54 \\
                        work location &          19 &      143 &          0 &       11 &   132 &  105 &   38 \\
                             religion &          31 &      209 &          0 &       11 &   198 &  173 &   36 \\
                                genre &          21 &      226 &          4 &       11 &   211 &  110 &  116 \\
                           instrument &          22 &      234 &          4 &       40 &   190 &  122 &  112 \\
                             employer &          29 &      174 &          0 &        0 &   174 &  104 &   70 \\
                              country &          43 &      261 &          0 &        4 &   257 &  147 &  114 \\
                      native language &          17 &       36 &          0 &        1 &    35 &    2 &   34 \\
                headquarters location &          10 &       90 &          6 &        6 &    78 &    0 &   90 \\
                           capital of &          14 &      182 &         84 &       14 &    84 &    0 &  182 \\
                          instance of &          21 &      420 &          2 &       68 &   350 &    0 &  420 \\
                location of formation &          17 &      272 &         32 &       62 &   178 &    0 &  272 \\
          twinned administrative body &          11 &       86 &         22 &        9 &    55 &   24 &   62 \\
                            continent &          15 &      120 &          0 &       18 &   102 &   50 &   70 \\
                    country of origin &          29 &      812 &          6 &      172 &   634 &    0 &  812 \\
                             location &          30 &      843 &          0 &      160 &   683 &   27 &  816 \\
                              capital &          14 &      182 &         80 &       22 &    80 &    0 &  182 \\
\bottomrule
\end{tabular}

}
    \caption{Elaborated stats of patterns in the \resource{}.}
    \label{tab:rel-graph-stats-elaborate}
\end{table*}

\section{Paraphrases Analysis}
\label{sec:paraphrase_analysis}

We analysis the type of paraphrases in \resource{}. Thus, we sample 100 paraphrase pairs from the agreement study that were in agreement with our annotation and label the type of paraphrase phenomena.
We mainly rely on a subset of paraphrase types defined by \citet{what_is_paraphrase}, but also define some new types which were not covered by that work.
We begin by briefly defining the types of paraphrases found in \resource{} from \citet{what_is_paraphrase} (more thorough definitions can be found in their paper), and then define the new types we observed.

\begin{table*}[t!]
% \small
    \centering
\resizebox{1\textwidth}{!}{%
\begin{tabular}{lllllllll}
\toprule
Paraphrase Type & Pattern \#1 & Pattern \#2 & Relation & N. \\
\midrule

Synonym substitution & [X] died in [Y]. & [X] expired at [Y]. & place of death & 41 \\

Function words variations & [X] is [Y] citizen. & [X], who is a citizen of [Y]. & country of citizenship & 16 \\

Converse substitution & [X] maintains diplomatic relations with [Y]. & [Y] maintains diplomatic relations with [X]. & diplomatic relation & 10 \\

Change of tense & [X] is developed by [Y]. & [X] was developed by [Y]. & developer & 10 \\ 

Change of voice & [X] is owned by [Y]. & [Y] owns [X]. & owned by & 7 \\

Verb/Noun conversion & The headquarter of [X] is in [Y]. & [X] is headquartered in [Y]. & headquarters location & 7 \\

External knowledge & [X] is represented by music label [Y]. & [X], that is represented by [Y]. & record label & 3 \\

Noun/Adjective conversion & The official language of [X] is [Y]. & The official language of [X] is the [Y] language. & official language & 2 \\

Change of aspect & [X] plays in [Y] position. & playing as an [X], [Y] & position played on team & 1 \\

\midrule

Irrelevant addition & [X] shares border with [Y]. & [X] shares a common border with [Y]. & shares border with & 11 \\

Topicalization transformation & [X] plays in [Y] position. & playing as a [Y], [X] & position played on team & 8 \\

Apposition transformation & [X] is the capital of [Y]. & [Y]'s capital, [X]. & capital of & 4 \\ 

Other syntactic movements & [X] and [Y] are twin cities. & [X] is a twin city of [Y]. & twinned administrative body & 10 \\








\bottomrule
\end{tabular}


}
\caption{Different types of paraphrases in \resource{}. We report examples from each paraphrase type, along with the type of relation, and the number of examples from the specific transformation from a random subset of 100 pairs. Each pair can be classified into more than a single transformation (we report one for brevity), thus the sum of transformation is more than 100.}
\label{tab:paraphrases_analysis}
    
    % \vspace{-0.1in}
\end{table*}


\begin{enumerate}
    \item Synonym substitution: Replacing a word/phrase by a synonymous word/phrase, in the appropriate context, results in a paraphrase of the original sentence/phrase.
    % \item Antonym substitution: Replacing a word/phrase by its antonym accompanied by a negation or by negating some other word, in the appropriate context, results in a paraphrase of the original sentence/phrase.
    \item Function word variations: Changing the function words in a sentence/phrase without affecting its semantics, in the appropriate context, results in a paraphrase of the original sentence/phrase.
    
    \item Converse substitution: Replacing a word/phrase with its converse and inverting the relationship between the constituents of a sentence/phrase, in the appropriate context, results in a paraphrase of the original sentence/phrase, presenting the situation from the converse perspective.
    \item Change of tense: Changing the tense of a verb, in the appropriate context, results in a paraphrase of the original sentence/phrase.
    \item Change of voice: Changing a verb from its active to passive form and vice versa results in a paraphrase of the original sentence/phrase.
    
    % \item Actor/Action substitution: Replacing the name of an action by a word/phrase denoting the person doing the action (actor) and vice versa, in the appropriate context, results in a paraphrase of the original sentence/phrase.
    % \item Verb/“Semantic-role noun” substitution: Replacing a verb by a noun corresponding to the agent of the action or the patient of the action or the instrument used for the action or the medium used for the action, in the appropriate context, results in a paraphrase of the original sentence/phrase.
    % \item Part/Whole substitution: Replacing a part by its corresponding whole and vice versa, in the appropriate context, results in a paraphrase of the original sentence/phrase.
    \item Verb/Noun conversion: Replacing a verb by its corresponding nominalized noun form and vice versa, in the appropriate context, results in a paraphrase of the original sentence/phrase.
    % \item Verb/Adjective conversion: Replacing a verb by the corresponding adjective form and vice versa, in the appropriate context, results in a paraphrase of the original sentence/phrase.
    % \item Verb/Adverb conversion: Replacing a verb by its corresponding adverb form and vice versa, in the appropriate context, results in a paraphrase of the original sentence/ phrase.
    \item External knowledge: Replacing a word/phrase by another word/phrase based on extra-linguistic (world) knowledge, in the appropriate context, results in a paraphrase of the original sentence/phrase.
    \item Noun/Adjective conversion: Replacing a verb by its corresponding adjective form and vice versa, in the appropriate context, results in a paraphrase of the original sentence/phrase.
    % \item Verb-preposition/Noun substitution: Replacing a verb and a preposition denoting location by a noun denoting the location and vice versa, in the appropriate context, results in a paraphrase of the original sentence/phrase.
    
    \item Change of aspect: Changing the aspect of a verb, in the appropriate context, results in a paraphrase of the original sentence/phrase.
    % \item Change of modality: Addition/deletion of a modal or substitution of one modal by another, in the appropriate context, results in a paraphrase of the original sentence/phrase.
    
\end{enumerate}

We also define several other types of paraphrases, which were not covered by \citet{what_is_paraphrase}, potentially as such cases did not appear in the corpora they have inspected.

\begin{enumerate}[a.]
\item Irrelevant addition: an addition or removal of a word or phrase, that does not affect the meaning of the sentence, and can be inferred from the context independently.
\item Topicalization transformation: a transformation from or to a topicalization construction. Topicalization is a construction in which a clause is moved to the beginning of a sentences.
\item Apposition transformation: a transformation from or to an apposition construction. In an apposition construction, two noun phrases where one identifies the other are placed one next to each other.
\item Other syntactic movements: includes other types of syntactic transformations that are not part of the other categories. This include cases such as moving an element from a coordinate construction to the subject position as in the last example in Table \ref{tab:paraphrases_analysis}. Another type of transformation is in the following paraphrase: ``[X] plays in [Y] position.'' and ``[X] plays in the position of [Y].'' where a compound noun-phrase is replaced with prepositional phrase.
\end{enumerate}


We report the percentage of each type, along with examples of paraphrases in Table \ref{tab:paraphrases_analysis}.

\resource{} can be found here: \url{https://github.com/yanaiela/pararel/tree/main/data/pattern_data/graphs_json}.