% This must be in the first 5 lines to tell arXiv to use pdfLaTeX, which is strongly recommended.
\pdfoutput=1
% In particular, the hyperref package requires pdfLaTeX in order to break URLs across lines.

\documentclass[11pt]{article}

% Remove the "review" option to generate the final version.
\usepackage[]{naacl2021}

% Standard package includes
\usepackage{times}
\usepackage{latexsym}
\usepackage{amsmath}
% For proper rendering and hyphenation of words containing Latin characters (including in bib files)
\usepackage[T1]{fontenc}
% For Vietnamese characters
% \usepackage[T5]{fontenc}
% See https://www.latex-project.org/help/documentation/encguide.pdf for other character sets

% This assumes your files are encoded as UTF8
\usepackage[utf8]{inputenc}

% This is not strictly necessary, and may be commented out,
% but it will improve the layout of the manuscript,
% and will typically save some space.
\usepackage{microtype}

% If the title and author information does not fit in the area allocated, uncomment the following
%
%\setlength\titlebox{<dim>}
%
% and set <dim> to something 5cm or larger.

\usepackage{soul} % colors
% \usepackage[dvipsnames,table,xcdraw]{xcolor}


% own packages
\usepackage{graphicx} % figures
% \usepackage{float}
\usepackage{subfig} % for subfloats
% \usepackage[caption=false]{subfig}
\usepackage{booktabs} % nice tables
\usepackage{multirow} % multi column
\usepackage{comment} % for commenting text
\usepackage{amssymb} % check mark
\usepackage{multicol}
\usepackage{dsfont} % math




\usepackage{transparent} % transparent

\definecolor{WildStrawberry}{HTML}{ffab91}
\definecolor{OrangeRed}{HTML}{ff7043}

\definecolor{LightBlue}{HTML}{98F5FF}


\DeclareRobustCommand{\hltrue}[1]{{\sethlcolor{LightBlue}\hl{#1}}}
\DeclareRobustCommand{\hlfalseo}[1]{{\sethlcolor{pink}\hl{#1}}}
\DeclareRobustCommand{\hlfalset}[1]{{\sethlcolor{WildStrawberry}\hl{#1}}}
\DeclareRobustCommand{\hlfalsetr}[1]{{\sethlcolor{OrangeRed}\hl{#1}}}


% \DeclareRobustCommand{\hlgray}[1]{{\sethlcolor{lightgray}\hl{#1}}}
\DeclareMathOperator*{\argmin}{argmin}
\DeclareMathOperator*{\argmax}{argmax}


\newcommand{\ye}[1]{\textcolor{purple}{Yanai: #1}}
\newcommand{\sr}[1]{\textcolor{blue}{Shauli: #1}}
\newcommand{\nk}[1]{\textcolor{brown}{Nora: #1}}
\newcommand{\ar}[1]{\textcolor{magenta}{Lasha: #1}}
\newcommand{\hs}[1]{\textcolor{orange}{Hinrich: #1}}
\newcommand{\yg}[1]{\textcolor{purple}{Yoav: #1}}
\newcommand{\am}[1]{\textcolor{red}{[Amit: #1]}}

\newcommand{\resource}{\textsc{ParaRel}\raisebox{-2pt}{\includegraphics[width=0.15in]{figures/horns}}}

\newcommand{\subj}{\textit{X}}
\newcommand{\obj}{\textit{Y}}

\title{Assessing Consistency of Language Models}
\title{Consistency in Language Models}
\title{(In)Consistency in Pretrained Language Models}
\title{Measuring and Improving Consistency in Pretrained Language Models}

% Author information can be set in various styles:
% For several authors from the same institution:
% \author{Author 1 \and ... \and Author n \\
%         Address line \\ ... \\ Address line}
% if the names do not fit well on one line use
%         Author 1 \\ {\bf Author 2} \\ ... \\ {\bf Author n} \\
% For authors from different institutions:
% \author{Author 1 \\ Address line \\  ... \\ Address line
%         \And  ... \And
%         Author n \\ Address line \\ ... \\ Address line}
% To start a seperate ``row'' of authors use \AND, as in
% \author{Author 1 \\ Address line \\  ... \\ Address line
%         \AND
%         Author 2 \\ Address line \\ ... \\ Address line \And
%         Author 3 \\ Address line \\ ... \\ Address line}


  
 \author{Yanai Elazar\textsuperscript{1,2} \,
 Nora Kassner\textsuperscript{3} \,
 Shauli Ravfogel\textsuperscript{1,2} \, 
 Abhilasha Ravichander\textsuperscript{4} \, \\
 {\bf Eduard Hovy\textsuperscript{4}\, 
 \bf Hinrich Sch\"utze\textsuperscript{3}\, 
 Yoav Goldberg\textsuperscript{1,2}}\\
\textsuperscript{1}Computer Science Department, Bar Ilan University \\
\textsuperscript{2}Allen Institute for Artificial Intelligence \\
\textsuperscript{3}Center for Information and Language Processing (CIS), LMU Munich\\
\textsuperscript{4}Language Technologies Institute, Carnegie Mellon University \\
  {\tt  \{yanaiela,shauli.ravfogel,yoav.goldberg\}@gmail.com}\\
  {\tt kassner@cis.lmu.de} 
  {\tt \{aravicha,hovy\}@cs.cmu.edu} 
  }
 

\newcounter{notecounter}
\newcommand{\enotesoff}{\long\gdef\enote##1##2{}}
\newcommand{\enoteson}{\long\gdef\enote##1##2{{
\stepcounter{notecounter}
{\large\bf
\hspace{1cm}\arabic{notecounter} $<<<$ ##1: ##2
$>>>$\hspace{1cm}}}}}
\enoteson
%\enotesoff

\begin{document}
\maketitle
\begin{abstract}
We study consistency in Pretrained Language Models (PLMs), with respect to factual knowledge. \textit{Consistency} --- that is, the ability to remain invariant to meaning-preserving alternations --- is a desired property of any model applied to natural language. Are Pretrained Language Models (PLMs) consistent?
For studying this property, we create a high-quality resource of cloze-style query English paraphrases, which we name \resource{}. It contains paraphrases for forty relations, with an overall of @@ paraphrases.
Using \resource{}, we show that all models we experiment with demonstrate poor consistency capabilities, though with high variance between relations.
We further analyze the representational spaces of these models, and analyze them to provide additional evidence of the poor structure, that suggests these models are currently not suitable for representing knowledge in a robust way.
Finally, we propose a method for improving consistency in these models and show promising results.

\end{abstract}


% \section{Introduction}
\label{sec:intro}

Pretrained Language Models (PLMs) have become popular in recent years and are used in many NLP tasks and applications.
An ongoing line of research is to understand how these models work, and analyzing what they can and cannot do.
For example, previous work showed that PLMs are very successful at capturing syntax, as was shown both in behavioral probes \cite{yoav-syntax} and structural probes \cite{structural-probe}.
On the other hand, other capabilities such as reasoning \cite{talmor2019olmpics}, commonsense \cite{Bosselut2019COMETCT,zhou2020evaluating} and world knowledge \cite{lama,jiang2020can} are moderate, and it has been questioned if these can be captured solely by PLMs at all \cite{Bender2020ClimbingTN}.


More recently, PLMs were suggested to be used as strong baselines for Knowledge Bases \cite{lama,jiang2020can}. Although the results of these models in the zero-shot setting outperform some supervised baselines, the results are still low, and the setup is limited (e.g. only reconstructing objects that consist of a single token, or the inability to return more than a single answer, or no answer whatsoever).
Yet, these studies lack an even more important property of KBs, that is \textit{consistency} and \textit{determinism}.
% \nk{I would add in this place a definition of the two terms}
% These properties require that the results returned from queries that are mapped into the same query in a KBs, will return the same results and in the same order.  
If two queries $q_1$ and $q_2$ are equivalent in the KB (e.g., if $q_1$ and $q_2$ are two equivalent ways to ask, in the formal language specified by the KB, for the retrieval of US presidents), they should return the same results, in the same order.
For example, consider the statement ```Homeland' was released on [MASK]'', where the `[MASK]' is the masked token that the model has to predict. We expect a model to predict the same answer with a pattern entailed by it: ``Homeland was aired on [MASK]'', e.g. \textit{Showtime}.
Note that in consistency, we do not expect the answers to be correct in the real world, which is a separate problem and has been addressed before, but for the predictions to be consistent, regardless of their true value.
These properties, although standard and well studied in classic KBs \cite{hansen2000probabilistic,Thimm:2009d,muino2011measuring}, are important for automatically constructed KBs.
% However, these properties were not studied in the context of LMs. 
However, it has not been studied to what extent PLMs used as KBs fulfill these requirements.
We note that consistency is largely \emph{independent} of accuracy, which is the property often studied: an inaccurate model can still err consistently, and it is possible that a specific PLM that is used as KBs is accurate \emph{only provided} it is queried in the ``correct" way, thus showing inconsistency when it is being used with semantically-equivalent natural-language queries.
Finally, these properties are important not only in the KBs usage, but in many other applications, such as Question Answering, Textual Entailment, etc, and was recently been addressed in the QA domain \cite{consistent-qa}. 


\begin{figure}[t!]
\centering

\includegraphics[width=1.\columnwidth]{figures/overview}

\caption{Overview of our approach. We begin with KB triplets (subject, pattern, object), of which we feed the (subject, pattern, [MASK]) into a PLM. 
We expect that a consistent model would predict the same answer for every two tuples of $(subject, pattern_1)$, $(subject, pattern_2)$ with an entailment connection between them. \nk{we should not use entailment but paraphrase connection right?}}
\label{fig:overview}
% \vspace{-6mm}
\end{figure}




In this work, we propose a new benchmark to test consistency in LMs, using factual knowledge that was claimed to be partially encoded in PLMs.
This benchmark includes a manually curated resource, that provides sets of similar patterns -- short textual prompts that describe some relation \sr{similar in what sense?}.
Then, expecting from consistent models to predict the same answer for every two patterns with an entailment edge. \sr{the term ``entailment edge" is undefined at this point. Further, it is not clear what entailment even mean at this point in the intro.}
% We do so by conditioning on some factual knowledge statements that are stored by the model (the premise), and test whether the model correctly predict an inferred statement (the hypothesis).
An overview of our approach is displayed in Figure \ref{fig:overview}.




In order to test these capabilities, we manually build an Entailment Graph \cite{berant2011global,berant2012global,javad2018learning,hosseini2019duality} for each of the 41 relations in the T-REx dataset \cite{trex} provided by LAMA \cite{lama}, such as: \textit{born-in}, \textit{is-a-citizen}, \textit{works-for}, etc. 
These graphs were built by experts, and provide a high-quality resource, which we name \resource{}.
% \nk{jump from one graph to multiple graphs} 
Each of these graphs contains between @@-@@ different nodes, where each node is a pattern, e.g. ``[X] was aired on [Y]'', where \textit{[X]} and \textit{[Y]} are slot fillers for a subject and object.
These graphs are directional, which represent the entailment direction between two patterns (e.g. ``[X] was premiered on [Y]'' entails ``[X] was aired on [Y]'', but not the other way around).
Moreover, each edge is also annotated with the modification type (e.g. syntactic or lexical). %\nk{do we need to talk about syntactic and lexical here}.
% Moreover, the graphs also con the type of inference, such as lexical inference or syntactic inference \sr{I don't know re the term ``syntactic/lexical inference". Maybe focus on the alternations: ``we annotate the alternations between any pair of patterns as syntactic or lexical".}, which allows us to test different kinds of inference capabilities.
Examples of edges of the graphs are displayed in Table \ref{tab:rel-graph-examples}.



% Our framework \nk{the distinction between our data and the introduced framework is not clear here.} enables to test for consistency across different types of alternations in PLMs \nk{alternation is very vague}, and in this work, enabled by \resource{} we test for two prominent and basic capabilities: consistency over lexical and syntactic alternations. However, the framework is more general and allows to test other types of inferences, such as commonsense, pragmatics, etc. %\ar{Where is the boundary between lexical inference and commonsense inference}
% Moreover, by dissecting the consistency of PLMs, we provide a method to disentangle their inference capabilities, with those which are learned during training on some inference dataset, for instance, MNLI \cite{mnli}. 
% This will allow to better understand what kinds of capabilities a model learns during the pretraining step, and what it learns from a finetuning dataset. \sr{this sentence is redundant}
By evaluating the zero-shot consistency properties of models, we also allow inspecting these capabilities out-of-the-box, without adding finetuning biases. This allows to better understand what kinds of consistencies a model learns during pretraining, and will allow to track and improve this property in future work.


% By combining the \resource{} with the proposed framework, we are able to test different PLMs and how strong their consistency capabilities are.
Using \resource{}, we are able to probe for consistency in multiple PLMs.
We find that overall, current models perform poorly on the consistency benchmark, although there is a high variance between the different relations. 

Finally, we propose a method to boost the consistency capabilities of models, by continuing the pretraining with an additional consistency loss. Our results show promising results and achieve better consistency performance, but there's still a big gap before achieving consistent models.

\section{Introduction}
\label{sec:intro}

Pretrained Language Models (PLMs) are
large neural networks that are
used in a wide variety of NLP tasks. They operate under a
pretrain-finetune paradigm: models are first \emph{pretrained} over a large text corpus and then \emph{finetuned} on a downstream task. PLMs are thought of as good language encoders, supplying basic language understanding capabilities that can be used with ease for many downstream tasks.

A desirable property of a good language understanding model
is \emph{consistency}: the ability of  making consistent
decisions in semantically equivalent contexts, reflecting a
systematic ability to generalize in the face of language variability.

%the ability to make consistent decisions, reflecting a systematic generalization ability to understand language, regardless of language variability. \sr{not so clear} %\sr{This sense of ``consistency" focuses on the ability to reason in a way that is invariant to meaning-preserving alternations. Note this differs from the way we initially thought about consistency: a model that always predicts the same answer is also ``consistent" but surely doesn't express this meaning of consistency. It's ok to focus on the meaning we focus on here, but if we do so, shouldn't we focus in the experiments on cases where the model initially predicted correctly? our current evaluation doesn't align, to my understanding, with this definition in the intro.}.  \ar{The definition of inconsistency in my mind, is having beliefs that result in a contradictions. The paraphrase thing is one way to test this for some applicable relations (if you believe 'X was born in Paris' and 'The birthplace of X is Delhi', this results in a contradiction). We can make this clearer.} 
Examples of consistency include: predicting the same answer in reading comprehension tasks regardless of paraphrase \cite{consistent-qa}; making consistent assignments in coreference resolution \cite{denis2009global,chang2011inference}; or making summaries factually consistent with  the original document \cite{kryscinski2020evaluating}.
While consistency is important in many tasks, nothing in the
training process explicitly targets it. One could hope that
the unsupervised training signal from large corpora
made available to PLMs such as BERT or RoBERTa
\cite{bert,roberta} is sufficient to induce consistency and
transfer it to downstream tasks.
%This property is important for many tasks involving language and is hard to obtain solely in a locally supervised setting. \sr{what is locally supervised? how do we know it's hard?} 
% Ideally, a PLM such as BERT or RoBERTa \cite{bert,roberta} would arrive with such capability, persist during the finetuning step and allow the new model to make consistent predictions. 
%Ideally, a PLM such as BERT or RoBERTa \cite{bert,roberta} would learn such capability during the pretraining phase and then transfer it to the downstream task. \sr{I'd present it as a question: is the unsupervised pre-training signal enough to \emph{induce} consistency?} 
In this paper, we show that this is not the case.
%\ar{I think we can be more explicit here of how consistency can act as evidence of a more general and systematic ability to understand language.}


\begin{figure}[t!]
\centering

\includegraphics[width=1.\columnwidth]{figures/overview}

\caption{Overview of our approach. We begin with KB tuples (\textit{subject}, \textit{object}), of which we fill into a relational \textit{pattern} the: (\textit{subject}, \textit{[MASK]}) tuple,
% \nk{() are off here, also not sure if "populated" pattern is clear. Maybe also mention that the pattern is an relational pattern.}, 
and feed it into a PLM. 
We expect that a consistent model would predict the same answer for every two patterns $pattern_1$, $pattern_2$ which are paraphrases. 
In the example above, the model makes an inconsistent prediction for the left part of the figure and a consistent prediction on right part, where the subject is different.}
\label{fig:overview}
% \vspace{-6mm}
\end{figure}


The recent rise of PLMs has sparked a discussion about whether these models can be used as Knowledge Bases (KBs) \cite{lama,petroni2020how,alpaqa,roberts2020much}. 
% \nk{I would also cite this one maybe: "How Much Knowledge Can You Pack Into the Parameters of a Language Model?"}. 
Consistency is a key property of KBs and is particularly important for automatically constructed KBs.
%However, an important property of KBs, especially in automatically constructed KBs, is consistency.
One of the biggest appeals of using a PLM as a KB is that we can
query it in natural language -- instead of relying on a specific schema.
The expectation is that PLMs abstract away from language and map queries in natural language into meaningful representations such that queries with identical intent but different language form yield the same answer. 
For example, the query ``\textit{Homeland} was released on [MASK]'' should produce the same answer as ``\textit{Homeland} was originally aired on [MASK]''.
Studying inconsistencies of PLM-KBs can also teach us about the organization of knowledge in the model or lack thereof. 
Finally, failure to behave in a consistent manner may point
to other representational issues
% \nk{other issues seems a bit vague. Which other ones beside antonyms and synonyms do you have in mind?},
such as the similarity between antonyms and synonyms
\cite{nguyen2016integrating}.
% , symmetricity of the representation \enote{hs}{explain symmetricity, give citation} and the inability to handle negation. \nk{there was a paper studying this in PLMs right. Shouldn't we cite that one here}

% \ar{should we say "consistency of factual knowledge in PLMS through invariance to paraphrasing? This sentence makes it seem to me that consistency is equivalent to invariance to paraphrasing, but I think consistency is a broader problem. }
% \ar{Can we change the binary question here. "Is the" -> "to what extent"}
In this work, we study the consistency of factual knowledge
in PLMs: Is the factual information we extract from PLMs
invariant to paraphrasing? We use zero-shot evaluation since
we want to inspect models directly, without adding biases
through finetuning. This allows us to assess how
much consistency was acquired during pretraining and to
compare the consistency of different models.

%\sr{what does ``progress" mean here?}. %\sr{Question: how do we separate between lack of consistency due to the extraction method (maybe other extraction method would yield more consistent predictions), and an inherent lack of consistency in the model's behavior?}


We introduce \resource{}, a new benchmark  measuring
consistency in PLMs by using factual knowledge that was found to be partially encoded in them (\S \ref{sec:probe}).
\resource{} is a manually curated resource
that provides patterns -- short textual prompts -- that are paraphrases of one another, with 328 paraphrases describing thirty-eight binary relations such as \textit{X born-in Y}, \textit{X works-for Y} (\S \ref{sec:rel-graph}).
We then test multiple PLMs for knowledge consistency, i.e., whether
a model  predicts the same answer for all patterns of a relation.
Figure \ref{fig:overview} shows an overview of our approach.
Using \resource{}, we probe for consistency in four PLM
types: BERT, BERT-whole-word-masking, RoBERTa and ALBERT (\S
\ref{sec:setup}).
Our experiments with \resource{} show that
current models have poor consistency although there is a high variance between different relations (\S \ref{sec:experiments}). 

Finally, we propose a method that improves model consistency
by introducing a novel consistency loss
(\S \ref{sec:adding_consistency}). We demonstrate that models trained with this
loss achieve better consistency
performance on new relations. However, there still is much
work to do to achieve fully consistent models.

% Extending upon previous work that showed that factual knowledge can be extracted to some degree \cite{lama,alpaqa}, we extend their proposed patterns that were used to extract that information, and manually write paraphrases to the original patterns.


% \resource{} contains 40 relations from the T-REx dataset \cite{trex} provided by LAMA \cite{lama}, such as: \textit{born-in}, \textit{is-a-citizen}, \textit{works-for}, etc (\S \ref{sec:rel-graph}).
% The paraphrases were built by experts, and provide a high-quality resource.
% \nk{jump from one graph to multiple graphs} 
% Each of these graphs contains between @@-@@ different nodes, where each node is a pattern, e.g. ``[X] was aired on [Y]'', where \textit{[X]} and \textit{[Y]} are slot fillers for a subject and object.
% Moreover, each edge is also annotated with the modification type (e.g. syntactic, lexical) \sr{I am not sure we need to mention these in the intro, given the pretty narrow / heuristic way we define those}.
% Examples of edges of the graphs are displayed in Figure \ref{fig:graph}.


%%%%%%%%%%%%%%%


% By combining the \resource{} with the proposed framework, we are able to test different PLMs and how strong their consistency capabilities are.



% Jonathan:
% https://www.aclweb.org/anthology/P12-1013.pdf

% https://www.aclweb.org/anthology/P11-1062.pdf

% https://www.aclweb.org/anthology/P10-1124.pdf


% Mohammad Javad Hosseini:
% https://www.aclweb.org/anthology/P19-1468.pdf

% https://www.mitpressjournals.org/doi/pdfplus/10.1162/tacl_a_00250

% inference
% https://arxiv.org/pdf/1909.07521.pdf
% https://arxiv.org/pdf/2002.05867.pdf
% https://arxiv.org/pdf/2006.06609.pdf
% https://arxiv.org/pdf/2007.00655.pdf

% \section{Consistency Probing of Knowledge}
\section{Probing PLMs for Consistency}
\label{sec:probe}

In this section, we formally define consistency and describe our framework for probing consistency of PLMs.

% In this section, we formally define consistency and describe our probing framework of PLMs consistency.


\subsection{Consistency}
We define a model as \emph{consistent} if, given  two
\textit{cloze-phrases} such as 
 ``\textit{Seinfeld} originally aired on \textit{[MASK]}'' and
``\textit{Seinfeld} premiered on \textit{[MASK]}'' that
are \textit{quasi-paraphrases}, it makes non-contradictory
predictions\footnote{We refer to \textit{non-contradictory
    predictions} as predictions that, as the name suggest,
  do not contradict one another. For instance, predicting as
  the birth place of a person two difference cities is
  considered to be contradictory, but predicting a city and
  its country, is \textbf{not}.}
%contradicting.}
on N-1 relations over a large set of entities.
A \textit{quasi-paraphrase} -- a concept introduced by \citet{what_is_paraphrase} -- is a more fuzzy version of a paraphrase. The concept does not rely on the strict, logical definition of paraphrase and allows to operationalize concrete uses of paraphrases. This definition is in the spirit of the RTE definition \cite{dagan-rte}, which similarly supports a more flexible use of the notion of entailment.
For instance, a model that predicts \textit{NBC} and \textit{ABC} on the two aforementioned patterns, is not consistent, since these two facts are contradictory. We define a \textit{cloze-pattern} as a cloze-phrase that expresses a relation between a subject and an object.
Note that consistency does not require the answers to be factually correct. While correctness is also an important property for KBs, we view it as a separate objective and measure it independently.
We use the terms \textit{paraphrase} and \textit{quasi-paraphrase} interchangeably.
 


Many-to-many (N-M) relations (e.g. \textit{shares-border-with}) can be consistent 
even with different answers (given they are correct). For
instance, two patterns that express the
\textit{shares-border-with} relation and predict
\textit{Albania} and \textit{Bulgaria} for
%the
\textit{Greece}
%subject
are both correct. We do not consider such relations for measuring consistency. However, another requirement from a KB is \textit{determinism}, i.e., returning the results in the same order (when more than a single result exists).
In this work, we focus on consistency, but also measure determinism of the models we inspect.

\subsection{The Framework}
\label{sec:framework}

\begin{figure*}[t!]
\centering

\includegraphics[width=1.\textwidth]{figures/framework}

\caption{Overview of our framework for assessing model
  consistency. $D_i$ (``Data Pairs $(D)$'' on the left) is a
  set of KB triplets of some relation $r_i$, which are
  coupled with a set of \textit{quasi-paraphrase}
  cloze-patterns $P_i$
(``Patterns $(P)$'' on the right)
  that describe that relation. We then populate the subjects
  from $D_i$ as well as a mask token into all patterns $P_i$
(shown in the middle)
  and expect a model to predict the same object across all pattern pairs.}
\label{fig:framework}
% \vspace{-6mm}
\end{figure*}


% \enote{hs}{``quasi-paraphrase cloze-pattern'': it would be
%   better to define clear terminology in the beginning and
%   then to use it consistently (!). it's confusing to the
%   reader to switch back and forth between terms for the same concept}


An illustration of the framework is presented in Figure \ref{fig:framework}.
Let $D_i$ be a set of subject-object KB tuples (e.g., <\textit{Homeland}, \textit{Showtime}>) from some relation $r_i$ (e.g., \textit{originally-aired-on}), accompanied with a set of \textit{quasi-paraphrases} cloze-patterns $P_i$ (e.g., \textit{X} originally aired on \textit{Y}).
Our goal is to test whether the model consistently predicts the same object (e.g., \textit{Showtime}) for a particular subject (e.g., \textit{Homeland}).\footnote{Although it is possible to also predict the subject from the object, in the cases of N-1 relations more than a single answer would be possible, making it impossible to test for consistency, but determinism.} To this end, we substitute \textit{X} with a subject from $D_i$ and \textit{Y} with \textit{[MASK]} in all of the patterns $P_i$ of that relation (e.g., \textit{Homeland} originally aired on \textit{[MASK]} and \textit{Homeland} premiered on \textit{[MASK]}).
A consistent model must predict the same entity. 


% \paragraph{Previous Description}
% Let $D = \{D_1, D_2, \dots, D_m\}$ be a set of sets of KB tuples, where each $D_i$ contains factual statements that express a specific relation $R_i$. In this section, we use $R_1$ = ``premiered on'' and $D_1$ as running examples. Each $D_i=\{d_1^i, d_2^i, \dots, d_n^i\}$ is composed of $n$ examples  where each $d_j^i = <s_j^i,o_j^i>$ is a subject-object tuple from the relation $R_i$. For example, $<\text{\textit{Homeland}}, \text{\textit{Showtime}}>$ is an example of a tuple in $D_1$. Each relation $R_i$ is associated with a set of cloze-patterns $P_i$ that are \textit{quasi-paraphrases} of each other, i.e., they express the same relation $R_i$. For example, $P_1=\{p_1^1, p_2^1, \dots, p_n^1\}$ contains the patterns ``\subj{} was originally aired on \obj{}'' and ``\subj{} premiered on \obj{}''. 


% Given some relation $R_i$, a subject-object tuple $d_j^i \in D_i$ (e.g., `Homeland', `Showtime') and two paraphrases $p_k^i, p_l^i \in P_i$ associated with this relation (such as ``\subj{} was originally aired on \obj{}'' and ``\subj{} premiered on \obj{}''), our goal is to test whether the model consistently predicts the same object for a particular subject. To this end, we substitute \subj{} with a selected subject and \obj{} with MASK; we write this as $p_k^i(s_j^i,mask)$, $p_l^i(s_j^i,mask)$ or (for a specific example) as ``Homeland was originally aired on [MASK]'', ``Homeland was premiered on [MASK]''. Finally, we extract the model's most probable prediction for MASK in the two sentences.  A consistent model must predict the same entity. Note that the consistency measure does not require the answers to be factually correct. While correctness is also an important property for KBs, we view it as a separate objective and measure it independently. 









\paragraph{Restricted Candidate Sets}

Since PLMs were not trained for serving as KBs, they often predict words that are not KB entities; e.g., a PLM may predict, for the pattern ``\textit{Showtime} originally aired on \textit{[MASK]}'', the noun `tv' --  which is also a likely substitution for the language modeling objective, but not a valid KB fact completion.
Therefore, following \citep{Xiong2020Pretrained, kassner2021multilingual}, we restrict the PLMs' output vocabulary to the set of possible gold objects for each relation from the underlining KB. For example, in the \textit{born-in} relation, instead of inspecting the entire vocabulary of a model, we only keep objects from the KB, such as \textit{Paris}, \textit{London}, \textit{Tokyo}, etc.

% I don't think we have to give implementation details at
% this point in the paper?
%In practice, we compute the full probability
%distribution and then only consider the subset of possible
%gold objects.

Note that this setup makes the task easier for the PLM,
especially in the context of KBs. However, poor
consistency in this setup strongly implies that consistency
would be even lower without restricting candidates.


\section{The \resource{} Resource}
\label{sec:rel-graph}


To test the consistency of PLMs, we opt to test their predictions sensitivity to variations patterns for which the same answer is expected. This requires a large set of paraphrased queries. To this end, we create a paraphrase resource that enables us to study this property, which we name \resource{}. \resource{} is a high-quality resource, built by experts. %, and results in a high-quality resource which we hope can contribute to others' work as well.
It contains patterns for 40 relations from the TREx dataset \cite{trex}, with an average of @@ patterns per relation.
Each pattern is represented as a node in an undirected graph, and each edge signifies the type of modification from one pattern to another (e.g. lexical change).
A schematic view of the ``aired on'' relation can be seen in Figure \ref{fig:graph}. 

% All patterns from a specific relation are paraphrases.
% Each pattern entails the main expression for each pattern, but not necessarily the other way around. 
% For example, for the relation of ``aired on'', which describes a TV-series that was aired on some network, the original pattern is: ``\subj{} was aired on \obj{}'' and one of the patterns we create is ``\subj{} was premiered on \obj{}'', which entails the main expression, but not vice versa.


% For instance, the phrase "\subj{} was aired on \obj{}" entails "\obj{} released \subj{}", but not "\subj{} was premiered on \obj{}", although the other direction is entailed.
% \ar{The entailment 'Y released X' doesn't seem correct. Suggestion For instance, the phrase "\subj{} was premiered on \obj{}" entails "\subj{} was aired on \obj{}", but not the other way around.}

\begin{figure}[t!]
\centering

\includegraphics[width=1.\columnwidth]{figures/ent_graph}

\caption{Overview of \resource{}.}
\label{fig:graph}
% \vspace{-6mm}
\end{figure}

% Overall, \resource{} contains @@ different patterns in total for @@ different relations (@@\% paraphrases per relation in average). 
% All of the paraphrases for a particular relation are organized as a directed graph to indicate the entailment relation between two patterns. 
% Each edge also contains information about the transformation type, e.g. a lexical or syntactic transformation (Examples for each can be seen in Figure \ref{fig:graph}).
%A detailed definition is provided shortly.
% \ye{need to define this.} 
% This information allows us to study the different generalization capabilities of models.
% The nodes of the graph (which account for the phrases), also contain additional information about the number of times that the pattern appeared in Wikipedia, @@ more?. %\ar{will we still have the wikipedia info in the paper?}

\resource{}, was constructed in four steps: (1) We begin
with the patterns provided by LAMA \cite{lama} (one pattern
per relation, referred to as base pattern). (2) Then, we
augmented that relation with other patterns that are
paraphrases of the base pattern. Some of these relations are
taken from LPAQA \cite{alpaqa}. However, since its relations
were extracted automatically, many do not accurately depict
the relation, therefore only a subset of these patterns were
used in practice.  (3) Then, by using spike \cite{spike}, we
searched for additional patterns that appeared in Wikipedia
and added them to our resource. This was done by searching
for subject and object tuples from the TREx resource and
looking at the sentences that contain both. (4) Lastly, we
add additional patterns that are paraphrases of the base
pattern using the annotators' linguistic
expertise.
(The annotators can also add entailing or entailed patterns,
but we do not make use of them in this paper.)
Then, two additional
authors went over all the patterns and corrected them, while
engaging in a discussion until reaching an agreement,
discarding patterns with disagreements.

% \nk{should we give a number?}.\ye{I don't think it's necessary}

% \nk{Not sure if the details of this need to be in the main paper or can be moved to the appendix?} \sr{I support moving the technicalities to an appendix and focus here on the properties of the resources and how it is being used.}
% \ye{can decide when we have the full story}
%The entailment edges were also annotated by the same two authors that created the patterns.

\begin{table}[t]
% \small
    \centering
\begin{tabular}{lr}
  \toprule
  % do we really need this?:
%         Property &  Value \\
%\midrule
    \# Relations &   38 \\
    \# Patterns & 328 \\
    \midrule
 Min \# patterns per rel. &    2 \\
 Max \# patterns per rel. &   20 \\
 Avg \# patterns per rel. &    8.63 \\
 \midrule
   Avg syntax  &    4.74 \\
  Avg lexical  &    6.03 \\
\bottomrule
\end{tabular}
    \caption{Statistics of \resource{}. 
    Last two rows: average number of unique
    syntactic/lexical variations of patterns for a relation.}
    \label{tab:rel-graph-stats}
    
    \vspace{-0.1in}
\end{table}



\paragraph{Consistency Types}
% \nk{we should change inference types to consistency types} \ye{but the graph is in entailment graph}
To study the consistency against syntactic and lexical variations of models, each edge is augmented with the type of changes performed to reach from one pattern to another. We account for three types of changes: 1) \textit{syntactic}, 2) \textit{lexical} and 3) \textit{determiner}, all binary variables that account for a change of the specific type between two patterns.
\textit{Syntactic} structure is defined as the dependency path between the subject and the object in a given pattern, where the path includes the edge types. Two patterns are considered equal syntactically if the labeled paths are identical.
\textit{Lexical} difference is defined by words difference between two patterns, excluding determiners, punctuation, and symbols. The addition or removal of a preposition does not count as a lexical change.
\textit{Determiner} difference is considered when the patterns' determiners do not fully overlap.


Some statistics of the graphs are presented in Table \ref{tab:rel-graph-stats}, and full statistics per relation are presented in Table \ref{tab:rel-graph-stats-elaborate} in the Appendix.



% \begin{table*}[t]
% \small
    \centering
\resizebox{1\textwidth}{!}{%
\begin{tabular}{llll}
\toprule
           Pattern & Base &  Entailed & Type \\
\midrule
% hi & byw & chao \\
      \textsc{Died-in} & \textsc{Subj} passed away in \textsc{Obj}. & $\Leftrightarrow$ \textsc{Subj} died in OBJ. & lexical \\
       & \textsc{Subj} was murdered at \textsc{Obj}. & $\Rightarrow$ \textsc{Subj} died in \textsc{Obj}. & lexical \& syntactic \\
       & \textsc{Subj} died at \textsc{Obj}. & $\Leftrightarrow$ \textsc{Subj} died in \textsc{Obj}. & lexical \\
       
       \midrule
       
       \textsc{Official-language} & \textsc{Obj} is the official language of \textsc{Subj}. & $\Leftrightarrow$ The official language of \textsc{Subj} is \textsc{Obj}. & syntactic \\
       & In \textsc{Subj} \textsc{Obj} is an official language. & $\Leftrightarrow$ The official language of \textsc{Subj} is \textsc{Obj}.	& \\
       
       \midrule
       
       \textsc{Located-in} & \textsc{Subj} is a county in \textsc{Obj}. & $\Rightarrow$ \textsc{Subj} is located in \textsc{Obj} . & lexical \& syntactic \\
       & \textsc{Subj} district, \textsc{Obj}.	& $\Rightarrow$ \textsc{Subj} is located in \textsc{Obj} . & lexical \& syntactic \\
       
       \midrule
       
       \textsc{Instrument} & \textsc{Subj} was a \textsc{Obj} player. & $\Rightarrow$ \textsc{Subj} plays \textsc{Obj}. & lexical \& syntactic \\
       & \textsc{Subj} played \textsc{Obj} . & $\Leftarrow$ \textsc{Subj} plays \textsc{Obj}. & tense \\
       & \textsc{Obj} sonatas of \textsc{Subj}. & $\Rightarrow$ \textsc{Subj} plays \textsc{Obj}. & lexical \& syntactic \\
       
\bottomrule
\end{tabular}
}
    \caption{Examples of patterns in the \resource{}.}
    \label{tab:rel-graph-examples}
\end{table*}



\section{Experimental Setup}
\label{sec:setup}

\subsection{Models}
We experiment with four PLM variants from three PLMs families: BERT \cite{bert}, BERT whole-word-masking, RoBERTa \cite{roberta} and ALBERT \cite{albert}, each with two of their  size variants.\footnote{For ALBERT we use the smallest and largest versions.}
% Additionally, we use the whole-word-masking version of BERT large, which was shown to perform well in multiple tasks \cite{talmor2019olmpics}.

In addition, we also report a majority baseline that always predicts the same most common object for a specific relation. By definition, this baseline is perfectly consistent.

\subsection{Data}


% To probe for consistency of PLMs we use cloze-style queries using a (subject, relation, object) triple from a KB. We then populate the subjects and objects from the KB into patterns for all triplets in $D$ from \resource{} to create the cloze-style queries; e.g, ``\subj{} was aired in \obj{}''. \subj{} is substituted with the subject and \obj{} with a masked token (e.g. `[MASK]'). \nk{not sure if this is a bit confusing as it is saying the same as earlier a bit differently}\yg{I agree. Don't repeat, and refer to prev description.}

We use knowledge graph data from TREx \cite{trex} which contains 34,039 tuples across 40 relations. To make the results comparable across all models, we remove objects that do not fit a single token in all models' vocabularies. Moreover, some relations contain aggregated data of different types, which makes the original LAMA pattern sometimes obsolete as it does not fit the type (e.g. the tuple (`My Family', `sitcom') in the \textit{genre} relation, that mainly contains music-genre entities).\yg{the prev sentence is not clear}\am{I agree} In these cases, we manually filter these entities from our test set.
Overall, after the filtering, we are left with @@ tuples.
% We use the patterns from our resource \resource{}, described in the previous section. \am{use them for, what?}


\subsection{Evaluation}
\label{sec:eval}

%\ye{I think typed querying should be defined properly}
% Following prior work \cite{Xiong2020Pretrained, }: for each relation, we create a
% candidate set $\mathcal{C}$ and then predict
% $\arg\max_{c\in \mathcal{C}}p(c|q)$ where $p(w|q)$ is the probability
% that word $w$ gets predicted in query $q$. For most relations,
% there is only one valid entity type, e.g., country for the ``captical-of'' relation.
% We choose as $\mathcal{C}$ the set of all possible objects for a specific relation from the TREx dataset.
% The candidate set could also be obtained from an entity
% typing system
% (e.g., \cite{yaghoobzadeh-schutze-2016-intrinsic}), but this
% is beyond the scope of this paper.


We compute the accuracy of predictions per relation, or graph, as the number of consistent predictions for every pattern pair for all triples in the KB for a relation, dividing by the total number of predictions.
\[
\sum_j \sum_{k,l} p_k^i(s_j^i,mask) = p_l^i(s_j^i,mask)
\]
\am{at first glance "=" looks like assignment to me}\am{missing denominator?}
Finally, we aggregate the results for all relations and report a final number as the average of all relations. This metric can be viewed as macro average.\am{I don't understand the metric}
% We also report aggregated results for specific edge types that involve a specific transformation: \textit{syntactic}, \textit{lexical}, and both, using the same measurement. \nk{what about det? Also in the statistics table you call it rest}



\section{Experiments and Results}
\label{sec:experiments}

% We report results on multiple splits of the data.
First, we report the overall consistency performances, per graph, as describe in Section \ref{sec:eval}.
The results for the different models are summarized in Table \ref{tab:entailment-main}.
% That is, the overall successful prediction of the entailed pattern, divided by the overall connection. This measurement can also be seen as a weighted average, since every edge gets a score averaged by the number of base patterns that were predicted correctly.
That is, the p1, 



\begin{table*}[t]
% \small
    \centering
\resizebox{1\textwidth}{!}{%
\begin{tabular}{lrrrrrrrr}
\toprule
                              Pattern &  bert-base &  bert-large &  bert-large-wwm &  roberta-base &  roberta-large &  albert-base &  albert-xxlarge &   n \\
\midrule
                          instance of &       0.40 &        0.47 &            0.44 &          0.30 &           0.38 &         0.35 &            0.36 &  21 \\
                             employer &       0.55 &        0.38 &            0.31 &          0.34 &           0.53 &         0.40 &            0.29 &   4 \\
               country of citizenship &       0.70 &        0.71 &            0.68 &          0.73 &           0.71 &         0.54 &            0.62 &  16 \\
                         record label &       0.32 &        0.26 &            0.29 &          0.33 &           0.35 &         0.35 &            0.48 &  37 \\
                                genre &       0.49 &        0.49 &            0.39 &          0.40 &           0.33 &         0.22 &            0.27 &  11 \\
                        position held &       0.60 &        0.48 &            0.46 &          0.40 &           0.33 &         0.50 &            0.34 &   6 \\
                    country of origin &       0.61 &        0.62 &            0.58 &          0.59 &           0.57 &         0.43 &            0.42 &  29 \\
                          subclass of &       0.75 &        0.76 &            0.79 &          0.70 &           0.73 &         0.69 &            0.75 &   4 \\
              applies to jurisdiction &       0.89 &        0.90 &            0.92 &          0.84 &           0.87 &         0.91 &            0.87 &   4 \\
                   shares border with &       0.54 &        0.54 &            0.56 &          0.38 &           0.47 &         0.45 &            0.55 &   9 \\
                              country &       0.83 &        0.84 &            0.83 &          0.77 &           0.77 &         0.74 &            0.73 &   3 \\
                    official language &       0.77 &        0.84 &            0.84 &          0.67 &           0.78 &         0.44 &            0.66 &  12 \\
                              capital &       0.72 &        0.78 &            0.81 &          0.50 &           0.67 &         0.59 &            0.76 &  14 \\
             language of work or name &       0.66 &        0.69 &            0.65 &          0.59 &           0.64 &         0.48 &            0.40 &  15 \\
                     original network &       0.25 &        0.28 &            0.29 &          0.27 &           0.30 &         0.26 &            0.25 &  28 \\
 position played on team / speciality &       0.29 &        0.31 &            0.32 &          0.32 &           0.36 &         0.49 &            0.56 &   9 \\
                headquarters location &       0.55 &        0.51 &            0.63 &          0.47 &           0.58 &         0.42 &            0.47 &  10 \\
                           instrument &       0.35 &        0.50 &            0.55 &          0.38 &           0.48 &         0.68 &            0.40 &  11 \\
                         manufacturer &       0.82 &        0.86 &            0.87 &          0.58 &           0.70 &         0.75 &            0.83 &  25 \\
          twinned administrative body &       0.67 &        0.69 &            0.66 &          0.50 &           0.54 &         0.50 &            0.57 &   8 \\
                            continent &       0.87 &        0.91 &            0.86 &          0.60 &           0.75 &         0.90 &            0.86 &   4 \\
                            developer &       0.61 &        0.60 &            0.77 &          0.64 &           0.73 &         0.55 &            0.69 &  16 \\
                              part of &       0.54 &        0.57 &            0.60 &          0.51 &           0.48 &         0.64 &            0.58 &   2 \\
                             owned by &       0.32 &        0.44 &            0.43 &          0.32 &           0.36 &         0.28 &            0.34 &   4 \\
                location of formation &       0.47 &        0.52 &            0.55 &          0.50 &           0.49 &         0.35 &            0.32 &  17 \\
                        field of work &       0.20 &        0.23 &            0.20 &          0.19 &           0.22 &         0.48 &            0.26 &  26 \\
                           capital of &       0.87 &        0.92 &            0.94 &          0.77 &           0.89 &         0.85 &            0.91 &  14 \\
                           occupation &       0.22 &        0.35 &            0.27 &          0.22 &           0.27 &         0.17 &            0.14 &  16 \\
                             location &       0.48 &        0.49 &            0.57 &          0.42 &           0.46 &         0.46 &            0.50 &  27 \\
                       place of death &       0.29 &        0.38 &            0.25 &          0.33 &           0.23 &         0.16 &            0.20 &   8 \\
                             religion &       0.29 &        0.34 &            0.46 &          0.34 &           0.32 &         0.19 &            0.36 &   3 \\
                       place of birth &       0.37 &        0.41 &            0.40 &          0.32 &           0.36 &         0.31 &            0.35 &  14 \\
                            member of &       0.79 &        0.85 &            0.82 &          0.80 &           0.87 &         0.83 &            0.81 &  15 \\
                        work location &       0.80 &        0.72 &            0.63 &          0.57 &           0.63 &         0.41 &            0.41 &   9 \\
                          named after &       0.76 &        0.71 &            0.75 &          0.54 &           0.64 &         0.66 &            0.73 &  20 \\
 original language of film or TV show &       0.86 &        0.90 &            0.80 &          0.64 &           0.67 &         0.63 &            0.52 &   6 \\
 \midrule
 Average & 0.57 & 0.59 & 0.59 & 0.49 & 0.54 & 0.50 & 0.52 & 13.25 \\
 
\bottomrule
\end{tabular}
}
    \caption{Consistency results for the different relations. Reporting the average accuracy on different LMs, along with the number of patterns per relation.}
    \label{tab:entailment-main}
\end{table*}

\begin{table*}[t]
% \small
    \centering
\resizebox{1\textwidth}{!}{%
\begin{tabular}{lrrrrrrr}
\toprule
{} &  bert-base &  bert-large &  bert-large-wwm &  roberta-base &  roberta-large &  albert-base &  albert-xxlarge \\
\midrule
syn   &             0.89 &              0.90 &                                 0.91 &          0.88 &           0.91 &            0.90 &               0.88 \\
lex   &             0.74 &              0.78 &                                 0.80 &          0.69 &           0.74 &            0.75 &               0.78 \\
both  &             0.85 &              0.83 &                                 0.89 &          0.81 &           0.84 &            0.88 &               0.86 \\ \midrule
uni   &             0.60 &              0.62 &                                 0.62 &          0.54 &           0.64 &            0.49 &               0.58 \\
bi    &             0.77 &              0.80 &                                 0.83 &          0.72 &           0.77 &            0.79 &               0.81 \\ \midrule
total &             0.75 &              0.78 &                                 0.81 &          0.70 &           0.76 &            0.77 &               0.79 \\
\bottomrule
\end{tabular}

}
    \caption{Consistent results aggregated on the different relations, by the different splits.}
    \label{tab:entailment-splits}
\end{table*}

\begin{table*}[t]
% \small
    \centering
\resizebox{1\textwidth}{!}{%
\begin{tabular}{llrrrrrrr}
\toprule
   type & index & bert-base &  bert-large &  bert-large-wwm &  roberta-base &  roberta-large &  albert-base &  albert-xxlarge \\

\midrule
\multirow{3}{*}{all} & 1-1 &             0.94 &              0.96 &                                 0.97 &          0.81 &           0.91 &            0.91 &               0.96 \\
   & N-1 &             0.78 &              0.81 &                                 0.82 &          0.73 &           0.77 &            0.80 &               0.80 \\
   & N-M &             0.53 &              0.57 &                                 0.54 &          0.39 &           0.52 &            0.43 &               0.59 \\
\cline{1-9}
\multirow{3}{*}{syntactic} & 1-1 &             0.89 &              0.96 &                                 0.97 &          0.30 &           0.74 &            0.84 &               0.95 \\
   & N-1 &             0.91 &              0.92 &                                 0.93 &          0.89 &           0.92 &            0.92 &               0.90 \\
   & N-M &             0.75 &              0.76 &                                 0.76 &          0.71 &           0.76 &            0.75 &               0.71 \\
\cline{1-9}
\multirow{3}{*}{lexical} & 1-1 &             0.94 &              0.96 &                                 0.97 &          0.81 &           0.91 &            0.91 &               0.95 \\
   & N-1 &             0.77 &              0.80 &                                 0.81 &          0.71 &           0.76 &            0.79 &               0.79 \\
   & N-M &             0.50 &              0.55 &                                 0.51 &          0.36 &           0.50 &            0.40 &               0.57 \\
\cline{1-9}
\multirow{3}{*}{both} & 1-1 &             0.96 &              0.98 &                                 0.98 &          0.86 &           0.93 &            0.93 &               0.97 \\
   & N-1 &             0.87 &              0.86 &                                 0.88 &          0.83 &           0.85 &            0.87 &               0.85 \\
   & N-M &             0.69 &              0.70 &                                 0.66 &          0.57 &           0.61 &            0.61 &               0.65 \\
\cline{1-9}
\multirow{3}{*}{uni} & 1-1 &            -1.00 &             -1.00 &                                -1.00 &         -1.00 &          -1.00 &           -1.00 &              -1.00 \\
   & N-1 &             0.72 &              0.71 &                                 0.70 &          0.67 &           0.74 &            0.68 &               0.71 \\
   & N-M &             0.60 &              0.71 &                                 0.60 &          0.37 &           0.52 &            0.63 &               0.65 \\
\cline{1-9}
\multirow{3}{*}{bi} & 1-1 &             0.94 &              0.96 &                                 0.97 &          0.81 &           0.91 &            0.91 &               0.96 \\
   & N-1 &             0.79 &              0.82 &                                 0.83 &          0.74 &           0.78 &            0.81 &               0.81 \\
   & N-M &             0.51 &              0.53 &                                 0.52 &          0.40 &           0.52 &            0.36 &               0.57 \\
\bottomrule
\end{tabular}

}
    \caption{Consistent results aggregated on the different relations, by the different splits.}
    \label{tab:entailment-splits}
\end{table*}

\begin{figure*}[t!]
\centering

\includegraphics[width=1.\textwidth]{figures/results}

\caption{Results summary, by relation, by split.}
\label{fig:resuls}
% \vspace{-6mm}
\end{figure*}

% \section{Experiments and Results}
\label{sec:experiments}

% We report results on multiple splits of the data.
First, we report the overall consistency performances, per graph, as describe in Section \ref{sec:eval}.
The results for the different models are summarized in Table \ref{tab:entailment-main}.
% That is, the overall successful prediction of the entailed pattern, divided by the overall connection. This measurement can also be seen as a weighted average, since every edge gets a score averaged by the number of base patterns that were predicted correctly.
That is, the p1, 



\begin{table*}[t]
% \small
    \centering
\resizebox{1\textwidth}{!}{%
\begin{tabular}{lrrrrrrrr}
\toprule
                              Pattern &  bert-base &  bert-large &  bert-large-wwm &  roberta-base &  roberta-large &  albert-base &  albert-xxlarge &   n \\
\midrule
                          instance of &       0.40 &        0.47 &            0.44 &          0.30 &           0.38 &         0.35 &            0.36 &  21 \\
                             employer &       0.55 &        0.38 &            0.31 &          0.34 &           0.53 &         0.40 &            0.29 &   4 \\
               country of citizenship &       0.70 &        0.71 &            0.68 &          0.73 &           0.71 &         0.54 &            0.62 &  16 \\
                         record label &       0.32 &        0.26 &            0.29 &          0.33 &           0.35 &         0.35 &            0.48 &  37 \\
                                genre &       0.49 &        0.49 &            0.39 &          0.40 &           0.33 &         0.22 &            0.27 &  11 \\
                        position held &       0.60 &        0.48 &            0.46 &          0.40 &           0.33 &         0.50 &            0.34 &   6 \\
                    country of origin &       0.61 &        0.62 &            0.58 &          0.59 &           0.57 &         0.43 &            0.42 &  29 \\
                          subclass of &       0.75 &        0.76 &            0.79 &          0.70 &           0.73 &         0.69 &            0.75 &   4 \\
              applies to jurisdiction &       0.89 &        0.90 &            0.92 &          0.84 &           0.87 &         0.91 &            0.87 &   4 \\
                   shares border with &       0.54 &        0.54 &            0.56 &          0.38 &           0.47 &         0.45 &            0.55 &   9 \\
                              country &       0.83 &        0.84 &            0.83 &          0.77 &           0.77 &         0.74 &            0.73 &   3 \\
                    official language &       0.77 &        0.84 &            0.84 &          0.67 &           0.78 &         0.44 &            0.66 &  12 \\
                              capital &       0.72 &        0.78 &            0.81 &          0.50 &           0.67 &         0.59 &            0.76 &  14 \\
             language of work or name &       0.66 &        0.69 &            0.65 &          0.59 &           0.64 &         0.48 &            0.40 &  15 \\
                     original network &       0.25 &        0.28 &            0.29 &          0.27 &           0.30 &         0.26 &            0.25 &  28 \\
 position played on team / speciality &       0.29 &        0.31 &            0.32 &          0.32 &           0.36 &         0.49 &            0.56 &   9 \\
                headquarters location &       0.55 &        0.51 &            0.63 &          0.47 &           0.58 &         0.42 &            0.47 &  10 \\
                           instrument &       0.35 &        0.50 &            0.55 &          0.38 &           0.48 &         0.68 &            0.40 &  11 \\
                         manufacturer &       0.82 &        0.86 &            0.87 &          0.58 &           0.70 &         0.75 &            0.83 &  25 \\
          twinned administrative body &       0.67 &        0.69 &            0.66 &          0.50 &           0.54 &         0.50 &            0.57 &   8 \\
                            continent &       0.87 &        0.91 &            0.86 &          0.60 &           0.75 &         0.90 &            0.86 &   4 \\
                            developer &       0.61 &        0.60 &            0.77 &          0.64 &           0.73 &         0.55 &            0.69 &  16 \\
                              part of &       0.54 &        0.57 &            0.60 &          0.51 &           0.48 &         0.64 &            0.58 &   2 \\
                             owned by &       0.32 &        0.44 &            0.43 &          0.32 &           0.36 &         0.28 &            0.34 &   4 \\
                location of formation &       0.47 &        0.52 &            0.55 &          0.50 &           0.49 &         0.35 &            0.32 &  17 \\
                        field of work &       0.20 &        0.23 &            0.20 &          0.19 &           0.22 &         0.48 &            0.26 &  26 \\
                           capital of &       0.87 &        0.92 &            0.94 &          0.77 &           0.89 &         0.85 &            0.91 &  14 \\
                           occupation &       0.22 &        0.35 &            0.27 &          0.22 &           0.27 &         0.17 &            0.14 &  16 \\
                             location &       0.48 &        0.49 &            0.57 &          0.42 &           0.46 &         0.46 &            0.50 &  27 \\
                       place of death &       0.29 &        0.38 &            0.25 &          0.33 &           0.23 &         0.16 &            0.20 &   8 \\
                             religion &       0.29 &        0.34 &            0.46 &          0.34 &           0.32 &         0.19 &            0.36 &   3 \\
                       place of birth &       0.37 &        0.41 &            0.40 &          0.32 &           0.36 &         0.31 &            0.35 &  14 \\
                            member of &       0.79 &        0.85 &            0.82 &          0.80 &           0.87 &         0.83 &            0.81 &  15 \\
                        work location &       0.80 &        0.72 &            0.63 &          0.57 &           0.63 &         0.41 &            0.41 &   9 \\
                          named after &       0.76 &        0.71 &            0.75 &          0.54 &           0.64 &         0.66 &            0.73 &  20 \\
 original language of film or TV show &       0.86 &        0.90 &            0.80 &          0.64 &           0.67 &         0.63 &            0.52 &   6 \\
 \midrule
 Average & 0.57 & 0.59 & 0.59 & 0.49 & 0.54 & 0.50 & 0.52 & 13.25 \\
 
\bottomrule
\end{tabular}
}
    \caption{Consistency results for the different relations. Reporting the average accuracy on different LMs, along with the number of patterns per relation.}
    \label{tab:entailment-main}
\end{table*}

\begin{table*}[t]
% \small
    \centering
\resizebox{1\textwidth}{!}{%
\begin{tabular}{lrrrrrrr}
\toprule
{} &  bert-base &  bert-large &  bert-large-wwm &  roberta-base &  roberta-large &  albert-base &  albert-xxlarge \\
\midrule
syn   &             0.89 &              0.90 &                                 0.91 &          0.88 &           0.91 &            0.90 &               0.88 \\
lex   &             0.74 &              0.78 &                                 0.80 &          0.69 &           0.74 &            0.75 &               0.78 \\
both  &             0.85 &              0.83 &                                 0.89 &          0.81 &           0.84 &            0.88 &               0.86 \\ \midrule
uni   &             0.60 &              0.62 &                                 0.62 &          0.54 &           0.64 &            0.49 &               0.58 \\
bi    &             0.77 &              0.80 &                                 0.83 &          0.72 &           0.77 &            0.79 &               0.81 \\ \midrule
total &             0.75 &              0.78 &                                 0.81 &          0.70 &           0.76 &            0.77 &               0.79 \\
\bottomrule
\end{tabular}

}
    \caption{Consistent results aggregated on the different relations, by the different splits.}
    \label{tab:entailment-splits}
\end{table*}

\begin{table*}[t]
% \small
    \centering
\resizebox{1\textwidth}{!}{%
\begin{tabular}{llrrrrrrr}
\toprule
   type & index & bert-base &  bert-large &  bert-large-wwm &  roberta-base &  roberta-large &  albert-base &  albert-xxlarge \\

\midrule
\multirow{3}{*}{all} & 1-1 &             0.94 &              0.96 &                                 0.97 &          0.81 &           0.91 &            0.91 &               0.96 \\
   & N-1 &             0.78 &              0.81 &                                 0.82 &          0.73 &           0.77 &            0.80 &               0.80 \\
   & N-M &             0.53 &              0.57 &                                 0.54 &          0.39 &           0.52 &            0.43 &               0.59 \\
\cline{1-9}
\multirow{3}{*}{syntactic} & 1-1 &             0.89 &              0.96 &                                 0.97 &          0.30 &           0.74 &            0.84 &               0.95 \\
   & N-1 &             0.91 &              0.92 &                                 0.93 &          0.89 &           0.92 &            0.92 &               0.90 \\
   & N-M &             0.75 &              0.76 &                                 0.76 &          0.71 &           0.76 &            0.75 &               0.71 \\
\cline{1-9}
\multirow{3}{*}{lexical} & 1-1 &             0.94 &              0.96 &                                 0.97 &          0.81 &           0.91 &            0.91 &               0.95 \\
   & N-1 &             0.77 &              0.80 &                                 0.81 &          0.71 &           0.76 &            0.79 &               0.79 \\
   & N-M &             0.50 &              0.55 &                                 0.51 &          0.36 &           0.50 &            0.40 &               0.57 \\
\cline{1-9}
\multirow{3}{*}{both} & 1-1 &             0.96 &              0.98 &                                 0.98 &          0.86 &           0.93 &            0.93 &               0.97 \\
   & N-1 &             0.87 &              0.86 &                                 0.88 &          0.83 &           0.85 &            0.87 &               0.85 \\
   & N-M &             0.69 &              0.70 &                                 0.66 &          0.57 &           0.61 &            0.61 &               0.65 \\
\cline{1-9}
\multirow{3}{*}{uni} & 1-1 &            -1.00 &             -1.00 &                                -1.00 &         -1.00 &          -1.00 &           -1.00 &              -1.00 \\
   & N-1 &             0.72 &              0.71 &                                 0.70 &          0.67 &           0.74 &            0.68 &               0.71 \\
   & N-M &             0.60 &              0.71 &                                 0.60 &          0.37 &           0.52 &            0.63 &               0.65 \\
\cline{1-9}
\multirow{3}{*}{bi} & 1-1 &             0.94 &              0.96 &                                 0.97 &          0.81 &           0.91 &            0.91 &               0.96 \\
   & N-1 &             0.79 &              0.82 &                                 0.83 &          0.74 &           0.78 &            0.81 &               0.81 \\
   & N-M &             0.51 &              0.53 &                                 0.52 &          0.40 &           0.52 &            0.36 &               0.57 \\
\bottomrule
\end{tabular}

}
    \caption{Consistent results aggregated on the different relations, by the different splits.}
    \label{tab:entailment-splits}
\end{table*}

\begin{figure*}[t!]
\centering

\includegraphics[width=1.\textwidth]{figures/results}

\caption{Results summary, by relation, by split.}
\label{fig:resuls}
% \vspace{-6mm}
\end{figure*}

\section{Analysis}
\label{sec:analysis}



\begin{figure}[t!]
\centering

\subfloat{\includegraphics[width=1.\columnwidth]{figures/P407_emb.pdf}}\\
%   \hfill
\subfloat{\includegraphics[width=1.\columnwidth]{figures/P36_emb.pdf}}
% \\


\caption{t-SNE plots for encoded patterns from the ``Language of work or name'' and ``Capital'' relations. Points of the same color represents same subject, Points of the same shape represents same pattern. A consistent representation should cluster based on subjects (colors) rather than patterns (shapes).}
\label{fig:tsne-emb}

\end{figure}

To provide additional insights on the way models operate on the patterns with the masked tokens they need to predict, we inspect their representations after encoding the cloze-style patterns.
We encode the patterns, populated with 50 random subjects along with the masked token, and inspect the final layer in the masked token index for all the paraphrases of multiple relations.
Then, we use t-SNE \cite{tsne} and present the results in Figure \ref{fig:tsne-emb}.
Each point represents a specific tuple, and are colored by the subjects, and the shape stands for the pattern.
A good representation would cluster the vectors together based on the subject, as then the predictions would more likely to be consistent. However, clusters based on patterns would suggest a worse encoding, since the subjects, which are of great importance in these paraphrases, are less taken into account in the representation.

We display the t-SNE figures for two relations, \textit{Language  of work or name} and \textit{Capital} in Figure \ref{fig:tsne-emb}.
It is clear that the first figure clusters mainly on the patterns, whereas the second figure clusters mainly on the subjects. These results are also consistent with the performance of these relations: @@\% and @@\%, which suggests a better representation for the later.
Additionally, we also perform clustering for the representations,\footnote{Using the KMeans algorithm.} once with the number of subjects and once with the number patterns, given as an oracle, hoping to fit the subjects or patterns clusters. Then, we calculate the v-measure metric for measuring the purity of each cluster.
A higher score on the subject-based would suggest a representation that better fits the purpose of a KB.
The v-measure results for these patterns are presented in Table \ref{tab:vmeasure-small}.
As expected, the pattern-based clustering is better for the \textit{language of work or name} relation is higher than the subject-based, and vice versa for the \textit{capital} relation.
The full clustering measures for all relations are reported in the Appendix.
\ar{Can we report correlation with performance for all relations here? Or say that in general we observe this trend?}
\begin{table}[t]
% \small
    \centering
% \resizebox{1\textwidth}{!}{%
\begin{tabular}{lrr}
\toprule
                                          Pattern &  pattern &  subject \\
\midrule
     language of work or name &            0.72 &            0.24 \\
                      capital &            0.41 &            0.62 \\
\bottomrule
\end{tabular}

% }
    \caption{V-measure clustering performance for the two relations. Reporting the results for clustering based on the pattern, and based on the subjects.}
    \label{tab:vmeasure-small}
\end{table}


\section{Improving Consistency in PLMs}
\label{sec:adding_consistency}

In the previous sections, we showed pretrained models are generally not consistent in their predictions, and previous works have noticed the lack of this property in a variety of downstream tasks.
% , and since PLMs are used for downstream tasks, lack of consistency is likely to affect them as well. 
% \nk{suggestion: cut: "lack of consistency is likely to affect them as well." , add: 
An ideal model would exhibit the consistency property after pretraining, and would then be able to transfer it to different downstream tasks. We therefore ask:
% Ideally, a consistent PLM would also reflect this property in downstream tasks.
Can we enhance current PLMs and make them more consistent?

\subsection{Consistency Improved PLMs}
% \nk{Maybe we need to emphasize already when introducing this that this is more of a case study rather than solving the issue}
We propose to improve the consistency of PLMs by continuing the pretraining step with a novel consistency loss. %\nk{Should we refer to the paper that does it in a similar way?}
We make use of the T-REx tuples and the paraphrases from \resource{}.

% \enote{hs}{below: why is the notation used here different from the rest of the paper?}
For each relation $r_i$, we
have a set of paraphrased patterns $P_i$ describing that relation.
%, each $P_j^i$ represents a paraphrase of some relation $R^i$.
We use a PLM to encode all patterns in $P_i$, after populating a subject that corresponds to the relation $r_i$ and a mask token. We expect the model to make the same prediction for the masked token for all patterns.
% Then we consider the probability distribution over the vocabulary of the masked word, denoted by $D^i_l$. Our goal is to make the distribution of every pair $D^i_l,D^i_m$ as close as possible.


\paragraph{Consistency Loss Function}
% Given two paraphrases from \resource{}, we expect the predictions of the masked tokens to be identical. However, as we showed in the previous sections, it is not the case. \nk{this is repeating the intro of section 8}
As we evaluate the model using acc@1, the straight-forward consistency loss would require these predictions to be identical:
\begin{gather*} 
\min_{\theta} \mbox{sim}(\arg\max_i f_\theta(P_n)[i], \arg\max_j f_\theta(P_m)[j])%\\
%s.t. Q_n = f(P_n; \theta), Q_m = f(P_m; \theta)
\end{gather*}
where $f_\theta(P_n)$ is the output of an encoding function (e.g., BERT) parameterized by $\theta$ (a vector) over input $P_n$, and $f_\theta(P_n)[i]$ is the score of the $i$th vocabulary item of the model.

% \enote{hs}{last bit: the score of the $i$th element of the
%   language models' input
%   vocabulary?}

%where $Q_n,Q_m$ are the vocabulary distributions for the masked token in the patterns $P_n,P_m$  and $f$ is the encoding function, e.g. BERT, parameterized by $\theta$.
% \nk{Is it necessary to talk about this option? Also we do that for the local training, right?}

However, this objective contains a comparison between the output of two argmax operations, making it discrete and discontinuous, and hard to optimize in a gradient-based framework. We instead relax the objective, and require that the predicted \emph{distributions} $Q_n = \mbox{softmax}(f_\theta(P_n))$, rather than the top-1 prediction, be identical to each other. %we use a softer constraint.
%Instead of optimizing for the same argmax, we optimize for the same distribution.
% We propose to enforce consistency, by requiring the distribution of the masked tokens to be identical. 
We use two-sided KL Divergence to measure similarity between distributions: $D_{KL}(Q_n^{r_i}||Q_m^{r_i}) + D_{KL}(Q_m^{r_i}||Q_n^{r_i})$
%the distributions $Q_n,Q_m$
%more similar. Since DKL is not a symmetric metric, we use a two-sided DKL for two patterns $P^i_m,P^i_n$:
%$D_{KL}(Q^i_n||Q^i_m) + D_{KL} (Q^i_m||Q^i_n)$
where $Q_n^{r_i}$ is the predicted distribution for pattern $P_n$ of relation $r_i$.

% \enote{hs}{above: is this simply the jensen shannon divdergence?}

As most of the vocabulary is not relevant for the
predictions, we filter it down to the $k$ tokens from the candidate set of each
relation (\S\ref{sec:framework}). We want to
maintain the original capabilities of the
model -- focusing on the candidate set helps to achieve this goal since most of the vocabulary is not affected by our new
loss.

% From preliminary results, we noticed that using only the candidate set is beneficial.

% We typically have multiple paraphrases; we use all of them, which can help in enforcing a more general solution. 
To encourage a more general solution, we make use of all the paraphrases together, and enforce all predictions to be as close as possible.
% Moreover, to achieve a more general solution, we use all paraphrases per relation, and optimize for a consistent 
Thus, the consistency loss for all pattern pairs for a particular relation $r^i$ is:
\[
\mathcal{L}_{c} = \sum_{n=1}^k \sum_{m=n+1}^k D_{KL}(Q^{r_i}_n||Q^{r_i}_m) + D_{KL}(Q^{r_i}_m||Q^{r_i}_n)
\]


\paragraph{MLM Loss}
Since the consistency loss is different from the
Cross-Entropy loss the PLM is trained on, we find it
important to continue the MLM loss on text data, similar to previous work \cite{geva2020injecting}.

We consider two alternatives for continuing the pretraining objective: (1) MLM on Wikipedia and (2) MLM on the patterns of the relations used for the consistency loss. We found that the latter works better. We denote this loss by $\mathcal{L}_{MLM}$.


\paragraph{Consistency Guided MLM Continual Training}

Combining our novel consistency loss with
the regular MLM loss, we continue the PLM training by
combining the two losses. The combination of the two losses
is determined by a hyperparameter $\lambda$, resulting in
the following final loss function:
\[
\mathcal{L} = \lambda \mathcal{L}_c + \mathcal{L}_{MLM}
\]
This loss is computed per relation, for one KB tuple. We have many of these instances, which we require to behave similarly. Therefore, we batch together $l=8$ tuples from the same relation and apply the consistency loss function to all of them.



% \paragraph{Relations used for training}
% The TREx relations contain different types and contain
% many location-related relations.

\subsection{Setup}


Since we evaluate our method on unseen relations, we also
split train and test by relation type (e.g., location-based relations, which are very common
in T-REx).  Moreover, our method is aimed to be simple,
effective, and to require only minimal supervision. Therefore,
we opt to use only three relations:
\textit{original-language}, \textit{named-after}, and
\textit{original-network}; these were chosen randomly, out of
the non-location related relations.\footnote{Many
  relations are location-based -- no training on them prevents train-test leakage.} 
For validation, we randomly pick three
relations of the remaining relations and use the remaining
twenty-five for testing.

We perform minimal tuning of the parameters \yenew{($\lambda \in {0.1, 0.5, 1}$)} to pick the best model, train for 3 epochs, and select the best model based on  \textit{Consistent-Acc} on the validation set.
For efficiency reasons, we use the base version of BERT.


% \nk{People would not understand why we exclude
% locations. I would cut this?}.

\subsection{Improved Consistency Results}

\begin{table}[t]
% \small
    \centering
\resizebox{1\columnwidth}{!}{%
\begin{tabular}{lrrr}
\toprule
Model &        Accuracy & Consistency & Consistent-Acc \\
\midrule


majority   &  24.4+-22.5 &  100.0+-0.0 &  24.4+-22.5 \\
\midrule
BERT-base  &  45.6+-27.6 &  58.2+-23.9 &  27.3+-24.8 \\
BERT-ft    &  \textbf{\textul{47.4}}+-27.3 &  \textbf{64.0}+-22.9 &  \textbf{\textul{33.2}}+-27.0 \\
\quad -consistency &  46.9+-27.6 &  60.9+-22.6 &  30.9+-26.3 \\
\quad -typed     &  46.5+-27.1 &  62.0+-21.2 &  31.1+-25.2 \\
\quad -MLM       &  16.9+-21.1 &  \textul{80.8}+-27.1 &   9.1+-11.5 \\

\bottomrule
\end{tabular}

}

    \caption{Knowledge and consistency results for the baseline, BERT base, and our model. 
    % We report the \textit{Accuracy} using the original patterns from LAMA, the \textit{Consistency}, and \textit{Consistency-Acc} metrics. 
    The results are average over the 25 test relations. Bolded numbers signifies best performance between the baseline and the finetuned model, underline signifies the best performance including the ablations.}
    \label{tab:consistency-ft}
    
    \vspace{-0.2in}
\end{table}

The results are presented in Table
\ref{tab:consistency-ft}. We report aggregated results
for the 25 relations in the test.
We again
report macro average (mean over relations) and
standard deviation.  We report the results of the majority
baseline (first row),   BERT-base  (second row)
and our new model (BERT-ft, third row).  First, we note
that our model significantly improves consistency:
to 64.0\%
(compared with 58.2\% for BERT-base,
an increase
of 5.8 points).  \textit{Accuracy} also improves compared to BERT-base, from 45.6\% to 47.4\%. Finally, and most
importantly, we see an increase of 5.9 points in
\textit{Consistent-Acc}, which is achieved due to the improved
consistency of the model.  Notably, these improvements
arise from training on merely three relations, meaning that
the model improved its consistency ability and generalized
to new relations.  We measure the statistical
significance of our method compared to the BERT baseline,
using McNemar's test (following
\citet{dror2018hitchhiker,dror2020statistical}) and find all
results to be significant ($p \ll 0.01$).

% \nk{I would emphasize here again that it was able to transfer consistency from the seen to the unseen relations and we should maybe add that this is not only syntax-based (is it?)}

We also perform an ablation study to quantify the utility of
the different components. First, we report on the finetuned
model without the consistency loss
(-consistency). Interestingly, it does improve over the
baseline (BERT-base), but it lags behind our finetuned model.
Second, applying our loss on the candidate set rather than
on the entire vocabulary is beneficial (-typed). Finally, by
not performing the MLM training on the generated patterns
(-MLM), the consistency results improve significantly
(80.8\%); however, this also hurts  \textit{Accuracy} and \textit{Consistent-Acc}.
MLM training seems to serve as a regularizer
that prevents catastrophic forgetting.


Our ultimate goal is to improve consistency in PLMs for better performance on downstream tasks. Therefore, we also experiment with finetuning on SQuAD \cite{squad}, and evaluating on paraphrased questions from SQuAD \cite{squad-paraphrase} using our consistency model. However, the results perform on par with the baseline model, both on SQuAD and the paraphrase questions. More research is required to show that consistent PLMs can also benefit downstream tasks.


\section{Discussion}
\label{sec:discussion}

\paragraph{Consistency for Downstream Tasks}

The rise of PLMs has improved many tasks, but has also brought a lot of expectations. The standard usage of these models is by pretraining on a large corpus of unstructured text and then finetune on a task of interest. The first step is thought of as proving a good language-understanding component, whereas the second step is used to teach the format and the nuances of a downstream task.

As discussed earlier, consistency is a crucial component of many NLP system \cite{du2019consistent,consistent-qa,denis2009global,kryscinski2020evaluating} and obtaining this skill from a PLM would be extremely beneficial and have the potential to make specialized consistency solutions in downstream tasks redundant.
Indeed, there is an ongoing discussion about the ability to acquire understanding of ``meaning" from raw text signal alone \cite{bender2020climbing}.
% the PLMs capabilities, that are trained solely on form (texts) \cite{bender2020climbing} \sr{``meaning capabiltiies" is unclear. ``The ability to acquire understanding of ``meaning" from raw text signal alone"? but generally I think this paragraph doesn't say much, and the quote of Bender can be incorporated in the intro.}.
Thus, our new benchmark will allow track the progress of consistency in PLMs.


\paragraph{Broader Sense of Consistency}
In this work we focus on one type of consistency, that is, consistency to paraphrases, however, the consistency term is broader than that.
For instance, previous work has studied the effect of negation on factual statements, which can also be seen as consistency \cite{Ettinger_2020,kassner-schutze-2020-negated}. As such, a consistent model is expected to return a different answer to the prompts: ``\textit{Birds} can \textit{[MASK]}'' and ``\textit{Birds} cannot \textit{[MASK]}''. The inability to do so, as was shown in these works, also shows the lack of models' consistency.


\paragraph{Usage of PLMs as KBs}
Our work follows the setup of \citet{lama,alpaqa}, where PLMs are being tested as KBs. While it is an interesting setup for probing models for knowledge and consistency, it lacks important properties of standard KBs: (1) the ability to return more than a single answer and (2) the ability to return no answer.
Although some heuristics can be used for allowing a PLM to do so, e.g. using a threshold on the probabilities, it is not the way that the model was trained, and thus may not be optimal.
As such, newer approaches propose to use PLMs as a starting point to more complex systems, that provide promising results and solve the above problems \cite{thorne2020neural}.


\paragraph{Brittleness of Neural Models}
Our work also relates to the problem of brittleness in neural networks. One example of this brittleness is the vulnerability to adversarial attacks \cite{adversarial_attacks,jia2017adversarial}.
The other problem, closer to the problem we explore in this work, is the poor generalization to paraphrases.
For example, \citet{squad-paraphrase} created a paraphrase version for a subset of SQuAD \cite{squad}, and showed that models' performance drops significantly. 
\citet{ribeiro2018semantically} proposed another method for creating paraphrases and performed a similar analysis for visual question answering and sentiment analysis. Recently, \citet{ribeiro-etal-2020-beyond} proposed \textsc{CheckList}, a system that tests models' vulnerability to several linguistic perturbations.
% In this work, we show that also the PLMs are susceptible to small perturbations, and thus, finetuning on some downstream task (and dataset), that typically are not extensive, and do not contain equivalent examples, are not likely to perform better with this regard.

\resource{} enables us to study the brittleness of PLMs, and separate between facts which are robustly encoded in the model, compared to mere `guesses', which may arise from some heuristic or spurious correlations with certain patterns \cite{poerner2020bert}. In practice, we show that PLMs are susceptible to small perturbations, and thus, finetuning on some downstream task (and dataset), that typically are not extensive, and do not contain equivalent examples, are not likely to perform better with this regard.


% In this work, we show that also the PLMs are susceptible to small perturbations, and thus, finetuning on some downstream task (and dataset), that typically are not extensive, and do not contain equivalent examples, are not likely to perform better with this regard.

% In this work, we are also able to separate between facts which are robustly encoded in the model, compared to mere `guesses', which may arise from some heuristic or spurious correlations with certain patterns \cite{poerner2020bert}.

% Finally, the ability to be consistent with respect to multiple ways of expressing the same meaning, indicates on the robustness of a model \sr{isn't it what the previous paragraph is saying?}. As such, a single pattern that a model succeeds on, cannot truly indicate on the knowledge that a model possess, but can come from other reasons such as spurious correlations, memorization, etc. A recent and related approach for behavioral testing of NLP models is \textsc{CheckList} that tests models' vulnerability to several linguistic perturbations \cite{ribeiro-etal-2020-beyond}.

\section{Conclusions}
\label{sec:conclusions}

In this work, we study the consistency of PLMs with regard to their ability to extract knowledge.
We build a high-quality resource named \resource{}, that contain @@ patterns for 40 different relations.
Using \resource{}, we measure consistency in multiple PLMs, including BERT, RoBERTa, and ALBERT, and show that although the two latter are superior in other tasks over BERT, they fall short in terms of consistency. However, overall the consistency ability of these models is low.
We release \resource{} along with data tuples from \cite{trex} as a new benchmark, to track the consistency of models to knowledge.
Finally, we propose a new and simple method to improve the consistency of PLMs, by continuing the pretraining step with a novel loss. We show this method to be effective and improves both the consistency of models as well as their ability to extract the correct facts.


% \section{Related Work}
\cite{schmitt-schutze-2019-sherliic}
\cite{yanaka-etal-2020-neural}
\cite{goodwin-etal-2020-probing}
\cite{kassner-schutze-2020-negated}
\cite{ettinger-2020-bert}
\cite{heinzerling2020language}

% \input{40_memorization}


% \section*{Acknowledgements}
% Tomer Wolfson, Ido Dagan
% for annotations: Alon Jacovi, Ori Shapira, Arie Cattan, Elron Bandel

% Entries for the entire Anthology, followed by custom entries
\bibliography{custom}
\bibliographystyle{acl_natbib}

% \appendix

% \section{Example Appendix}
% \label{sec:appendix}

% This is an appendix.

\section{Appendix}
\label{sec:appendix}

We heavily rely on Hugging Face's Transformers library \cite{wolf-etal-2020-transformers} for all experiments involving the PLMs.
We used Weights \& Biases for tracking and logging the experiments \cite{wandb}.
Finally, we used sklearn \cite{scikit-learn} for other ML-related experiments.

% \begin{table*}[t]
% \small
    \centering
\resizebox{1\textwidth}{!}{%
\begin{tabular}{lrrrrrrr}
\toprule
                                index &  n\_patterns &  n\_edges &  syntactic &  lexical &  both &  uni &   bi \\
\midrule
                        field of work &          26 &      650 &         16 &       98 &   536 &    0 &  650 \\
                           occupation &          16 &      240 &          0 &       40 &   200 &    0 &  240 \\
                          named after &          21 &      381 &         44 &       35 &   302 &    0 &  381 \\
                       place of birth &          14 &      182 &          4 &       20 &   158 &    0 &  182 \\
                       place of death &          12 &      110 &          0 &       31 &    79 &   20 &   90 \\
                        position held &          28 &       57 &          0 &        7 &    50 &   51 &    6 \\
                     original network &          37 &     1050 &         20 &       77 &   953 &  258 &  792 \\
                   shares border with &          16 &      140 &         20 &       19 &   101 &   68 &   72 \\
 position played on team / speciality &           9 &       72 &         28 &        8 &    36 &    0 &   72 \\
             language of work or name &          18 &      228 &         12 &       14 &   202 &   12 &  216 \\
                    official language &          12 &      132 &         18 &        2 &   112 &    0 &  132 \\
 original language of film or TV show &          21 &      285 &         12 &       62 &   211 &  117 &  168 \\
                         manufacturer &          25 &      600 &         16 &       90 &   494 &    0 &  600 \\
                            developer &           0 &        0 &          0 &        0 &     0 &    0 &    0 \\
                  diplomatic relation &          19 &      342 &         70 &       50 &   222 &    0 &  342 \\
                             owned by &          23 &      141 &          0 &       25 &   116 &   87 &   54 \\
                        work location &          19 &      143 &          0 &       11 &   132 &  105 &   38 \\
                             religion &          31 &      209 &          0 &       11 &   198 &  173 &   36 \\
                                genre &          21 &      226 &          4 &       11 &   211 &  110 &  116 \\
                           instrument &          22 &      234 &          4 &       40 &   190 &  122 &  112 \\
                             employer &          29 &      174 &          0 &        0 &   174 &  104 &   70 \\
                              country &          43 &      261 &          0 &        4 &   257 &  147 &  114 \\
                      native language &          17 &       36 &          0 &        1 &    35 &    2 &   34 \\
                headquarters location &          10 &       90 &          6 &        6 &    78 &    0 &   90 \\
                           capital of &          14 &      182 &         84 &       14 &    84 &    0 &  182 \\
                          instance of &          21 &      420 &          2 &       68 &   350 &    0 &  420 \\
                location of formation &          17 &      272 &         32 &       62 &   178 &    0 &  272 \\
          twinned administrative body &          11 &       86 &         22 &        9 &    55 &   24 &   62 \\
                            continent &          15 &      120 &          0 &       18 &   102 &   50 &   70 \\
                    country of origin &          29 &      812 &          6 &      172 &   634 &    0 &  812 \\
                             location &          30 &      843 &          0 &      160 &   683 &   27 &  816 \\
                              capital &          14 &      182 &         80 &       22 &    80 &    0 &  182 \\
\bottomrule
\end{tabular}

}
    \caption{Elaborated stats of patterns in the \resource{}.}
    \label{tab:rel-graph-stats-elaborate}
\end{table*}



\end{document}
