\section{Experimental Setup}
\label{sec:setup}

\subsection{Models \& Data}
\label{setupdata}
We experiment with four PLMs:
%variants:
BERT, BERT whole-word-masking\footnote{BERT whole-word-masking is BERT's version where words that are tokenized into multiple tokens are masked together.}
\cite{bert}, RoBERTa \cite{roberta} and ALBERT
\cite{albert}. For BERT, RoBERTa and ALBERT, we use a base and a large version.\footnote{For ALBERT we use the smallest and largest versions.}
% \footnote{We use Transformers \cite{wolf-etal-2020-transformers} for all experiments involving the PLMs, Weights \& Biases \cite{wandb} for tracking and logging the experiments and sklearn \cite{scikit-learn} for other ML-related experiments.}
% Additionally, we use the whole-word-masking version of BERT large, which was shown to perform well in multiple tasks \cite{talmor2019olmpics}.
We also report a majority baseline that always predicts the most common object for a relation. By definition, this baseline is perfectly consistent.

% \subsection{Data}


% To probe for consistency of PLMs we use cloze-style queries using a (subject, relation, object) triple from a KB. We then populate the subjects and objects from the KB into patterns for all triplets in $D$ from \resource{} to create the cloze-style queries; e.g, ``\subj{} was aired in \obj{}''. \subj{} is substituted with the subject and \obj{} with a masked token (e.g. `[MASK]'). \nk{not sure if this is a bit confusing as it is saying the same as earlier a bit differently}\yg{I agree. Don't repeat, and refer to prev description.}

We use knowledge graph data from T-REx \cite{trex}.\footnote{We discard three poorly defined relations from T-REx.} To make the results comparable across models, we remove objects that are not represented as a single token in all models'
vocabularies; 26,813 tuples remain.\footnote{In a few cases, we filter entities from certain relations that contain multiple fine-grained relations to make our patterns compatible with the data. For instance, most of the instances for the \textit{genre} relation describes music genres, thus we remove some of the tuples were the objects include non-music genres such as `satire', `sitcom' and `thriller'.}
We further split the data into N-M relations for which we
report  determinism results (seven relations) and N-1
relations for which we report consistency (thirty-one relations). 
% Moreover, some relations contain aggregated data of different types, which makes the original LAMA pattern sometimes obsolete as it does not fit the type (e.g. the tuple (`My Family', `sitcom') in the \textit{genre} relation, that mainly contains music-genre entities).\yg{the prev sentence is not clear}\am{I agree} In these cases, we manually filter these entities from our test set.
% We use the patterns from our resource \resource{}, described in the previous section. \am{use them for, what?}

% \enote{hs}{above: i don't understand the footnote}

\subsection{Evaluation}
\label{sec:eval}

%\ye{I think typed querying should be defined properly}
% Following prior work \cite{Xiong2020Pretrained, }: for each relation, we create a
% candidate set $\mathcal{C}$ and then predict
% $\arg\max_{c\in \mathcal{C}}p(c|q)$ where $p(w|q)$ is the probability
% that word $w$ gets predicted in query $q$. For most relations,
% there is only one valid entity type, e.g., country for the ``captical-of'' relation.
% We choose as $\mathcal{C}$ the set of all possible objects for a specific relation from the TREx dataset.
% The candidate set could also be obtained from an entity
% typing system
% (e.g., \cite{yaghoobzadeh-schutze-2016-intrinsic}), but this
% is beyond the scope of this paper.

Our consistency measure for a relation $r_i$ (\textit{Consistency}) is the
percentage of consistent predictions of all the pattern pairs
$p_k^i,p_l^i \in P_i$ of that relation , for all its KB tuples
$d^i_j \in D_i$. Thus, for each KB tuple from a relation
$r_i$ that contains $n$ patterns, we consider predictions
for $n(n-1)/2$ pairs.
%adding this here: "corresponding to the relations."
%would imply that there are also members of D_i that
%do not correspond to the relation. but that is not the case?

% \enote{hs}{old-fashioned people prefer to use the word
%   ``metric'' only for metrics}

We also report \textit{Accuracy}, that is, the acc@1 of a model in predicting the correct object, using the original patterns from \citet{lama}. In contrast to \citet{lama}, we define it as the accuracy of the top-ranked object from the candidate set of each relation.
Finally, we report \emph{Consistent-Acc}, a new measure that evaluates individual objects as correct only
if \emph{all} patterns of the corresponding relation predict the object
correctly. \textit{Consistent-Acc} is  much stricter  and combines the
requirements of both consistency (\textit{Consistency}) and factual correctness (\textit{Accuracy}).

% In all our metrics, we report the average results over all relations, which can be viewed as a macro average.

We report the average over relations, i.e., macro average.
