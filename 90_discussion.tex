\section{Discussion}
\label{sec:discussion}

\paragraph{Consistency for Downstream Tasks}

The rise of PLMs has improved many tasks but has also brought a lot of expectations. The standard usage of these models is pretraining on a large corpus of unstructured text and then finetuning on a task of interest. The first step is thought of as providing a good language-understanding component, whereas the second step is used to teach the format and the nuances of a downstream task.

As discussed earlier, consistency is a crucial component of many NLP systems \cite{du2019consistent,consistent-qa,denis2009global,kryscinski2020evaluating} and obtaining this skill from a PLM would be extremely beneficial and have the potential to make specialized consistency solutions in downstream tasks redundant.
Indeed, there is an ongoing discussion about the ability to acquire understanding of ``meaning'' from raw text signal alone \cite{bender2020climbing}.
% the PLMs capabilities, that are trained solely on form (texts) \cite{bender2020climbing} \sr{``meaning capabilities" is unclear. ``The ability to acquire understanding of ``meaning" from raw text signal alone"? but generally I think this paragraph doesn't say much, and the quote of Bender can be incorporated in the intro.}.
Our new benchmark makes it possible to track the progress of consistency in PLMs.


\paragraph{Broader Sense of Consistency}
In this work we focus on one type of consistency, that is,
consistency in the face of paraphrasing; however, consistency is
a broader concept.  For instance, previous work has studied
the effect of negation on factual statements, which can also
be seen as consistency
\cite{Ettinger_2020,kassner-schutze-2020-negated}. 
A consistent model is expected to return  different answers
to the prompts: ``\textit{Birds} can \textit{[MASK]}'' and
``\textit{Birds} cannot \textit{[MASK]}''. The inability to
do so, as was shown in these works, also shows the lack of
model consistency.


\paragraph{Usage of PLMs as KBs}
Our work follows the setup of \citet{lama,alpaqa}, where PLMs are being tested as KBs. While it is an interesting setup for probing models for knowledge and consistency, it lacks important properties of standard KBs: (1) the ability to return more than a single answer and (2) the ability to return no answer.
Although some heuristics can be used for allowing a PLM to do so, e.g., using a threshold on the probabilities, it is not the way that the model was trained, and thus may not be optimal.
Newer approaches that propose to use PLMs as a starting point to more complex systems have promising results and address these problems \cite{thorne2020neural}.

\yenew{
Another approach, by \citet{autoprompt} suggests to use \textsc{AutoPrompt} to automatically generate prompts, or patterns, instead of creating them manually. This approach is superior to manual patterns \cite{lama}, or aggregation of patterns that were collected automatically \cite{alpaqa}. 
}

\paragraph{Brittleness of Neural Models}
Our work also relates to the problem of the brittleness of neural networks. One example of this brittleness is the vulnerability to adversarial attacks \cite{adversarial_attacks,jia2017adversarial}.
The other problem, closer to the problem we explore in this work, is the poor generalization to paraphrases.
For example, \citet{squad-paraphrase} created a paraphrase version for a subset of SQuAD \cite{squad}, and showed that model performance drops significantly. 
\citet{ribeiro2018semantically} proposed another method for
creating paraphrases and performed a similar analysis for
visual question answering and sentiment analysis. Recently,
\citet{ribeiro-etal-2020-beyond} proposed
\textsc{CheckList}, a system that tests a model's vulnerability to several linguistic perturbations.
% In this work, we show that also the PLMs are susceptible to small perturbations, and thus, finetuning on some downstream task (and dataset), that typically are not extensive, and do not contain equivalent examples, are not likely to perform better with this regard.

\resource{} enables us to study the brittleness of PLMs, and
separate  facts that are robustly encoded in the model from
mere `guesses', which may arise from some heuristic or
spurious correlations with certain patterns
\cite{poerner2020bert}. We showed that PLMs are susceptible
to small perturbations, and thus, finetuning on a
downstream task -- given that training datasets  typically are not
large and  do not contain equivalent examples -- is not
likely to perform better with respect to brittleness.


% In this work, we show that also the PLMs are susceptible to small perturbations, and thus, finetuning on some downstream task (and dataset), that typically are not extensive, and do not contain equivalent examples, are not likely to perform better with this regard.

% In this work, we are also able to separate between facts that are robustly encoded in the model, compared to mere `guesses', which may arise from some heuristic or spurious correlations with certain patterns \cite{poerner2020bert}.

% Finally, the ability to be consistent with respect to multiple ways of expressing the same meaning, indicates the robustness of a model \sr{isn't it what the previous paragraph is saying?}. As such, a single pattern that a model succeeds on, cannot truly indicate the knowledge that a model possesses, but can come from other reasons such as spurious correlations, memorization, etc. A recent and related approach for behavioral testing of NLP models is \textsc{CheckList} that tests models' vulnerability to several linguistic perturbations \cite{ribeiro-etal-2020-beyond}.

\paragraph{Can We Expect from LMs to be Consistent?}

\yenew{The typical training procedure of a LM does not necessarily encourage consistency. The standard training solely tries to minimize the log-likelihood of an unseen token, and this objective is not always aligned with consistency to knowledge. Consider for example the case of wikipedia texts, as opposed to reddit; the texts and styles in the first, may be very different, and even describe contradictory facts. A model can exploit the styles of each text, to best fit the probabilities given to an unseen word, even if they contradict each other.}

\yenew{Since the pretraining-finetuning procedure is the dominating one in our field currently, a great amount of the language capabilities that were learned during the pre-training also propagates to the fine-tuned models. As such, we believe it is important to measure and improve consistency already in the pretrained models.}

\paragraph{Reasons Behind the (In)Consistency}

\yenew{Since LMs are not expected to be consistent, what are the reasons behind their prediction, when being consistent, or inconsistent?}

\yenew{In this work, we presented the predictions of multiple queries, and the representation space of one of the inspected models. However, this does not point to the origins of such behavior.
In future work, we aim to inspect this question more closely.}
