\section{Experiments and Results}
\label{sec:experiments}


\begin{table*}[t]
% \small
    \centering
\resizebox{1\textwidth}{!}{%
\begin{tabular}{lllllllll}
\toprule
Models &       LAMA & Group-Score &    Succ-Pat &  Succ-Objs &  Know-Const &   Unk-Const &  Const-Objs & Consistency \\

\midrule
majority-baseline &  24.2+-20.7 &  24.2+-20.7 &  100.0+-0.0 &  24.9+-20.7 &  100.0+-0.0 &  100.0+-0.0 &  100.0+-0.0 &  100.0+-0.0 \\ \midrule
bert-base         &  44.1+-25.6 &  25.0+-25.5 &  100.0+-0.0 &  60.6+-23.5 &  61.4+-23.2 &  45.2+-20.6 &  32.5+-28.5 &  57.3+-22.3 \\
bert-large        &  45.6+-27.0 &  27.5+-27.6 &   99.7+-1.8 &  63.0+-25.2 &  62.4+-22.4 &  46.4+-18.5 &  34.2+-30.0 &  59.1+-21.5 \\
bert-large-wwm    &  45.4+-26.4 &  27.2+-27.6 &  100.0+-0.0 &  61.6+-23.2 &  62.6+-23.6 &  46.5+-18.7 &  33.8+-29.8 &  58.6+-22.7 \\
roberta-base      &  36.7+-24.6 &  16.0+-19.8 &  100.0+-0.0 &  54.2+-24.5 &  53.3+-20.2 &  41.2+-15.3 &  21.7+-23.1 &  50.1+-18.4 \\
roberta-large     &  40.8+-26.1 &  21.3+-23.7 &  100.0+-0.0 &  58.5+-24.1 &  58.2+-21.1 &  44.4+-16.0 &  27.4+-25.6 &  54.8+-19.6 \\
albert-base       &  29.9+-24.2 &  17.0+-22.2 &   99.7+-1.8 &  45.3+-25.7 &  56.1+-21.5 &  43.1+-18.0 &  25.1+-25.4 &  51.2+-19.8 \\
albert-xxlarge    &  39.5+-26.1 &  21.9+-25.7 &   99.7+-1.8 &  56.3+-25.8 &  56.7+-22.7 &  38.8+-14.7 &  26.8+-26.8 &  51.5+-21.1 \\
\bottomrule
\end{tabular}


}
    \caption{Consistency results for the different models\am{maybe add midrules to separate bert, roberta, albert}}
    \label{tab:consistency_main}
\end{table*}

\subsection{Consistency}

The results for the different models are summarized in Table \ref{tab:consistency_main}.
% First, we report the average consistency results for all graphs, as describe in Section \ref{sec:eval}.
The average results, across all relations are shown in the last row of Table \ref{tab:consistency_main}. The BERT based models achieves the higher scores over all the models. Moreover, there is a consistent improvement from the base to large versions of each models, although it is not always significantly higher (e.g. 57.3\% to 59.1\% in BERT, and 50.1\% to 54.8\% in RoBERTa).
However, in contrast to many previous works that observed quantitative and qualitative improvements of RoBERTa-based models over BERT, in terms of consistency BERT is better than RoBERTa, and ALBERT.
Still, the overall results are remarkably low (59.1\% for the best model1), especially given the permissive evaluation that only looks at the candidate set.
We note that the results are highly variant between models, and relations. For instance, the best performing model on the `capital of' relation achieves 94\%, whereas the best performing model on `owned by' is 44\%. 
On the other hand, the models performance on the `original language of film or TV show' relation, varies between 52\% and 90\% accuracy.
We report the full consistency results for each relation in the Appendix.

\begin{table}[t]
% \small
    \centering
\resizebox{1\columnwidth}{!}{%
\begin{tabular}{lrrr}
\toprule
Model&        Acc & Consistency & Consistent-Acc \\

\midrule
majority               &  23.1+-21.0 &  100.0+-0.0 &  23.1+-21.0 \\
\midrule
RoBERTa-med-small-1M-1 &   11.2+-9.4 &  37.1+-11.0 &    2.8+-4.0 \\
\midrule
RoBERTa-base-10M-1     &  17.3+-15.8 &  29.8+-12.7 &    3.2+-5.1 \\
RoBERTa-base-100M-1    &  22.1+-17.1 &  31.5+-13.0 &    3.7+-5.3 \\
RoBERTa-base-1B-1      &  \textbf{38.0}+-23.4 &  \textbf{50.6}+-19.8 &  \textbf{18.0}+-16.0 \\
\bottomrule
\end{tabular}
}
    \caption{Average knowledge and consistency results for the different RoBERTas, trained on increasing amounts of data, over all relations, along with the Std.}
    \label{tab:robertas}
\end{table}

\paragraph{Number of Tokens Effect}
Next, we inspect the question of whether the number of tokens used during the pretraining contributes to consistency.
We use the pretrained RoBERTas model provided by \citet{robertas} and repeat the experiments on four additional models.
These models are RoBERTa-based models, that were trained on sample of Wikipedia and the book corpus, with varying training tokens, and models size. In practice, we use one of the three published models for each configuration, and report the average accuracy over the relations for each model in Table \ref{tab:robertas}.
Overall, the consistency and the knowledge scores (LAMA and Group-Score) improves when trained on more data. However, there is an interesting outlier to this trend.
First, the model that was trained on one million tokens is more consistent than the models trained on ten and one-hundred million tokens. A potentially crucial difference is that this model is also smaller in terms of the number of parameters than the rest, in order to avoid overfitting. This may be part of the explanation, however it is nonetheless interesting that a model which is trained on significantly less data can achieve better consistency. On the other hand, the knowledge scores are lower, arguably due to less factual knowledge this model was exposed to during training.

% Finally, the original RoBERTa that was trained on ten billion parameters, achieves the same consistency performance as the one trained on one billion tokens, suggesting on a limit to the benefit of training examples in pretraining. \nk{not sure about this as the table is not there yet, but shouldn't it be: Finally, the original RoBERTa that was trained on ten billion tokens, achieves the same consistency performance as the one trained on one billion tokens, suggesting on a limit to the benefit of large training corpora. }



% \begin{table*}[t]
% \small
    \centering
\resizebox{1\textwidth}{!}{%
\begin{tabular}{lrrrrrrr}
\toprule
{} &  bert-base &  bert-large &  bert-large-wwm &  roberta-base &  roberta-large &  albert-base &  albert-xxlarge \\
\midrule
syn   &             0.89 &              0.90 &                                 0.91 &          0.88 &           0.91 &            0.90 &               0.88 \\
lex   &             0.74 &              0.78 &                                 0.80 &          0.69 &           0.74 &            0.75 &               0.78 \\
both  &             0.85 &              0.83 &                                 0.89 &          0.81 &           0.84 &            0.88 &               0.86 \\ \midrule
uni   &             0.60 &              0.62 &                                 0.62 &          0.54 &           0.64 &            0.49 &               0.58 \\
bi    &             0.77 &              0.80 &                                 0.83 &          0.72 &           0.77 &            0.79 &               0.81 \\ \midrule
total &             0.75 &              0.78 &                                 0.81 &          0.70 &           0.76 &            0.77 &               0.79 \\
\bottomrule
\end{tabular}

}
    \caption{Consistent results aggregated on the different relations, by the different splits.}
    \label{tab:entailment-splits}
\end{table*}

% \begin{table*}[t]
% \small
    \centering
\resizebox{1\textwidth}{!}{%
\begin{tabular}{llrrrrrrr}
\toprule
   type & index & bert-base &  bert-large &  bert-large-wwm &  roberta-base &  roberta-large &  albert-base &  albert-xxlarge \\

\midrule
\multirow{3}{*}{all} & 1-1 &             0.94 &              0.96 &                                 0.97 &          0.81 &           0.91 &            0.91 &               0.96 \\
   & N-1 &             0.78 &              0.81 &                                 0.82 &          0.73 &           0.77 &            0.80 &               0.80 \\
   & N-M &             0.53 &              0.57 &                                 0.54 &          0.39 &           0.52 &            0.43 &               0.59 \\
\cline{1-9}
\multirow{3}{*}{syntactic} & 1-1 &             0.89 &              0.96 &                                 0.97 &          0.30 &           0.74 &            0.84 &               0.95 \\
   & N-1 &             0.91 &              0.92 &                                 0.93 &          0.89 &           0.92 &            0.92 &               0.90 \\
   & N-M &             0.75 &              0.76 &                                 0.76 &          0.71 &           0.76 &            0.75 &               0.71 \\
\cline{1-9}
\multirow{3}{*}{lexical} & 1-1 &             0.94 &              0.96 &                                 0.97 &          0.81 &           0.91 &            0.91 &               0.95 \\
   & N-1 &             0.77 &              0.80 &                                 0.81 &          0.71 &           0.76 &            0.79 &               0.79 \\
   & N-M &             0.50 &              0.55 &                                 0.51 &          0.36 &           0.50 &            0.40 &               0.57 \\
\cline{1-9}
\multirow{3}{*}{both} & 1-1 &             0.96 &              0.98 &                                 0.98 &          0.86 &           0.93 &            0.93 &               0.97 \\
   & N-1 &             0.87 &              0.86 &                                 0.88 &          0.83 &           0.85 &            0.87 &               0.85 \\
   & N-M &             0.69 &              0.70 &                                 0.66 &          0.57 &           0.61 &            0.61 &               0.65 \\
\cline{1-9}
\multirow{3}{*}{uni} & 1-1 &            -1.00 &             -1.00 &                                -1.00 &         -1.00 &          -1.00 &           -1.00 &              -1.00 \\
   & N-1 &             0.72 &              0.71 &                                 0.70 &          0.67 &           0.74 &            0.68 &               0.71 \\
   & N-M &             0.60 &              0.71 &                                 0.60 &          0.37 &           0.52 &            0.63 &               0.65 \\
\cline{1-9}
\multirow{3}{*}{bi} & 1-1 &             0.94 &              0.96 &                                 0.97 &          0.81 &           0.91 &            0.91 &               0.96 \\
   & N-1 &             0.79 &              0.82 &                                 0.83 &          0.74 &           0.78 &            0.81 &               0.81 \\
   & N-M &             0.51 &              0.53 &                                 0.52 &          0.40 &           0.52 &            0.36 &               0.57 \\
\bottomrule
\end{tabular}

}
    \caption{Consistent results aggregated on the different relations, by the different splits.}
    \label{tab:entailment-splits}
\end{table*}


% \begin{figure*}[t!]
% \centering

% \includegraphics[width=1.\textwidth]{figures/results}

% \caption{Results summary, by relation, by split.}
% \label{fig:resuls}
% % \vspace{-6mm}
% \end{figure*}

% \begin{table*}[t]
% \small
    \centering
\resizebox{1\textwidth}{!}{%
\begin{tabular}{lrrrrrrrr}
\toprule
                              pattern &  know &  unk &  top@1 &  group-score &  const-subjs &  succ-patt &  succ-subjs &  consistency \\
\midrule
                             religion &                      0.35 &                        0.22 &      0.12 &            0.09 &                 0.09 &                  1.0 &                 0.88 &         0.34 \\
                       place of death &                      0.43 &                        0.32 &      0.42 &            0.04 &                 0.05 &                  1.0 &                 0.54 &         0.38 \\
                              capital &                      0.91 &                        0.40 &      0.70 &            0.57 &                 0.61 &                  1.0 &                 0.75 &         0.78 \\
                           instrument &                      0.48 &                        0.62 &      0.50 &            0.03 &                 0.05 &                  1.0 &                 0.85 &         0.50 \\
                             employer &                      0.38 &                        0.38 &      0.10 &            0.03 &                 0.11 &                  1.0 &                 0.27 &         0.38 \\
                headquarters location &                      0.65 &                        0.35 &      0.36 &            0.21 &                 0.25 &                  1.0 &                 0.53 &         0.51 \\
                        position held &                      0.49 &                        0.46 &      0.29 &            0.08 &                 0.11 &                  1.0 &                 0.65 &         0.48 \\
                            continent &                      0.92 &                        0.74 &      0.90 &            0.83 &                 0.85 &                  1.0 &                 0.97 &         0.91 \\
                            member of &                      0.92 &                        0.27 &      0.84 &            0.72 &                 0.74 &                  1.0 &                 0.89 &         0.85 \\
                              country &                      0.89 &                        0.74 &      0.60 &            0.54 &                 0.77 &                  1.0 &                 0.65 &         0.84 \\
                          subclass of &                      0.83 &                        0.53 &      0.61 &            0.56 &                 0.63 &                  1.0 &                 0.78 &         0.76 \\
                          instance of &                      0.51 &                        0.19 &      0.53 &            0.05 &                 0.05 &                  1.0 &                 0.86 &         0.47 \\
                              part of &                      0.62 &                        0.53 &      0.47 &            0.31 &                 0.57 &                  1.0 &                 0.51 &         0.57 \\
                             owned by &                      0.55 &                        0.23 &      0.51 &            0.21 &                 0.22 &                  1.0 &                 0.65 &         0.44 \\
                         record label &                      0.26 &                        0.28 &      0.06 &            0.00 &                 0.00 &                  1.0 &                 0.79 &         0.26 \\
                        field of work &                      0.28 &                        0.21 &      0.15 &            0.01 &                 0.01 &                  1.0 &                 0.39 &         0.23 \\
                        work location &                      0.74 &                        0.68 &      0.42 &            0.28 &                 0.42 &                  1.0 &                 0.59 &         0.72 \\
             language of work or name &                      0.70 &                        0.60 &      0.72 &            0.21 &                 0.23 &                  1.0 &                 0.85 &         0.69 \\
          twinned administrative body &                      0.57 &                        0.70 &      0.04 &            0.02 &                 0.37 &                  1.0 &                 0.07 &         0.69 \\
                         manufacturer &                      0.88 &                        0.33 &      0.90 &            0.62 &                 0.62 &                  1.0 &                 0.96 &         0.86 \\
                location of formation &                      0.59 &                        0.50 &      0.11 &            0.03 &                 0.13 &                  1.0 &                 0.18 &         0.52 \\
                                genre &                      0.52 &                        0.41 &      0.61 &            0.03 &                 0.03 &                  1.0 &                 0.79 &         0.49 \\
                    official language &                      0.87 &                        0.70 &      0.64 &            0.49 &                 0.64 &                  1.0 &                 0.81 &         0.84 \\
 position played on team / speciality &                      0.30 &                        0.32 &      0.06 &            0.00 &                 0.00 &                  1.0 &                 0.45 &         0.31 \\
                            developer &                      0.63 &                        0.28 &      0.71 &            0.22 &                 0.22 &                  1.0 &                 0.91 &         0.60 \\
                             location &                      0.56 &                        0.34 &      0.49 &            0.01 &                 0.01 &                  1.0 &                 0.70 &         0.49 \\
                          named after &                      0.78 &                        0.47 &      0.60 &            0.38 &                 0.41 &                  1.0 &                 0.80 &         0.71 \\
                    country of origin &                      0.67 &                        0.54 &      0.40 &            0.20 &                 0.25 &                  1.0 &                 0.63 &         0.62 \\
                     original network &                      0.29 &                        0.25 &      0.33 &            0.00 &                 0.00 &                  1.0 &                 0.82 &         0.28 \\
                       place of birth &                      0.46 &                        0.38 &      0.26 &            0.00 &                 0.01 &                  1.0 &                 0.32 &         0.41 \\
              applies to jurisdiction &                      0.94 &                        0.69 &      0.76 &            0.74 &                 0.82 &                  1.0 &                 0.83 &         0.90 \\
                           occupation &                      0.34 &                        0.35 &      0.01 &            0.00 &                 0.00 &                  1.0 &                 0.14 &         0.35 \\
                           capital of &                      0.95 &                        0.51 &      0.89 &            0.80 &                 0.82 &                  1.0 &                 0.92 &         0.92 \\
 original language of film or TV show &                      0.89 &                        0.90 &      0.58 &            0.50 &                 0.79 &                  1.0 &                 0.64 &         0.90 \\
               country of citizenship &                      0.74 &                        0.65 &      0.36 &            0.27 &                 0.32 &                  1.0 &                 0.68 &         0.71 \\
                   shares border with &                      0.54 &                        0.54 &      0.26 &            0.06 &                 0.14 &                  1.0 &                 0.54 &         0.54 \\
\bottomrule
\end{tabular}
}
    \caption{Additional results for the BERT-large model. }
    \label{tab:bert-results}
\end{table*}


\subsection{Consistency and Knowledge}

To better understand the results, and assess the knowledge and ability of the different patterns to extract the correct knowledge of \resource{}, we report additional metrics that help understanding the results.

First, we report the number of triplets that were predicted correctly by at least one of the patterns (Succ-Objs), and the number of patterns that were useful in predicting the right object at least once (Succ-Patt). Succ-Objs quantifies the degree of which the knowledge is stored by the models, and Succ-Patt quantified the quality of the patterns to extract the knowledge.
Across models, all patterns were useful at least once in extracting the right answer, except for three models (bert-large and the two versions of albert) where one pattern in the `diplomatic relation' relation did not succeed to classify any tuple.
On the other hand, some tuples are not predicted correctly by any of the patterns, as is shown in the Succ-Objs measure. Since most relations contain at least @@ patterns, we take it as evidence that this knowledge is not stored in the model. 
% \nk{This seems to suggest that it is the fault of the pattern that knowledge is not extracted from the model but an unsuccessful prediction across all models of the same tuple rather says that this knowledge is not stored} 
The average number of extractable tuples is between 45.3\% to 63.0\%, depending on the model. Notably, the BERT based models are consistently better than the rest in extracting the knowledge. 

Next, we report LAMA results, that is the acc@1 accuracy of a model in predicting the object, using the original patterns from \citet{lama}. This metric inspect whether the first object within the candidate set of each relation is the correct answer. The reported numbers for this metric differ from \citet{lama} as we use a candidate set, as well as take only the KB triplet where the objects are a single token in all models we inspect. The results range between 29.9 with albert-base to 45.6 with bert-large, being the best model in that aspect.
Additionally, we report another metric, which we call \textsc{Group-Score}, where a point is given only when all patterns predict the object correctly. This metric is much stricter than \textit{acc@1}, however it provide a more robust metric, and better quantifies the degree to which the knowledge is encoded into the model. Overall, it combines both the requirement of a model to be consistent and correct. The results are much lower than the LAMA metric, as expected, but follow the same trend: albert-base perform worse, 17.0\% and bert-large performs best with 27.5\% accuracy.

Next, we measure the consistency results for the tuples subset where at least one of the patterns for a specific relation predicted the correct answer, as well as for the subset where no pattern predicted the right answer correctly, which we refer to these metrics as \textsc{Know-Const} and \textsc{Unk-Const}, respectively.
A higher score on the first metric would suggest that storing the correct answer is related to the consistency of the model, and performing consistent predictions across paraphrases.
Indeed, the results appear to support this claim. For instance, the Know-Const metric for bert-large is 62.4 and Unk-Const is 46.4. \nk{I would emphasise this even more and say that our graph is able to distinguish pattern dependent guessing and the other is knowledge stored in the model}
% point on a connection between acquiring the correct knowledge, and the ability to extract it robustly. 
% We refer to these metrics as \textsc{Know} and \textsc{Unk}, respectively, that stands for Knowledgeable and Unknowledgeable consistency.
The additional consistent measurement we report is Const-Objs, which measure the number of tuples that are predicted consistently across all patterns for a particular relation.
\ye{not sure if this one is important / alternatively maybe use this one instead of the main consistency score?}




% For brevity, we report the results solely on BERT-large, which achieved the highest results. The results can be viewed in Table \ref{tab:bert-results}.


An interesting observation from the results, is the clear superiority of the BERT models over the others, both in the factual knowledge correctness and consistency. This result, that we hypothesis to be due to the training data, may have broader impact on the models to come: Training bigger models on more data is not always beneficial. Since wikipedia is likely the largest source of factual knowledge that exists in unstructured data, the focused training on it (apart from the book corpus) was likely to make the BERT models memorize it better.
As such, models such as GPT-3 \cite{gpt3}, or other future that solely train on more data with more parameters are not likely to be more consistent and absorb more factual knowledge.
