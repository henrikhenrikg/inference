%We study consistency in Pretrained Language Models (PLMs),
%with respect to factual knowledge.

\textit{Consistency} of a model --- that is, the invariance
of its behavior under meaning-preserving alternations in its
input --- is a highly desirable property in natural language
processing.  In this paper we study the question: Are
Pretrained Language Models (PLMs) consistent with respect to
factual knowledge?\ar{Should we make this question less
binary- "and if not, to what extent are they consistent"}
\enote{hs}{my preference would be to keep it simple and
short in the abstract and explain the whole complexity of
the problem in the body of the paper}
To this end, we create \resource{}, a
high-quality resource of cloze-style query English
paraphrases. It contains a total of 328 paraphrases for thirty-eight relations. Using \resource{}, we show that the consistency
of all PLMs we experiment with is poor -- though with high
variance between relations.  Our analysis of the
representational spaces of PLMs suggests that they have a
poor structure and are currently not suitable for
representing knowledge in a robust way.  Finally, we propose
a method for improving model consistency and experimentally
demonstrate its effectiveness.
