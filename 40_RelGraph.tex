\section{The \resource{} Resource}
\label{sec:rel-graph}

We now describe \resource{}, a concrete resource according to the framework described in Section \ref{sec:framework}.
\resource{} is curated by experts, with high level of agreement.
It contains patterns for thirty-eight relations from the T-REx \cite{trex} --- a large dataset containing KB triples aligned with Wikipedia abstracts --- with an average of 8.63 patterns per relation.
Some statistics of the resource are presented in Table \ref{tab:rel-graph-stats}.


\paragraph{Construction Method}
\resource{} was constructed in four steps. (1) We began with
the patterns provided by LAMA \cite{lama} (one pattern per
relation, referred to as \textit{base-pattern}). (2) We augmented each base-pattern with other patterns that are paraphrases from
LPAQA \cite{alpaqa}. However, since LPAQA was
created automatically (either by back-translation, or by extracting patterns from sentences that contain both subject and object), some LPAQA patterns are not
correct paraphrases.
We therefore only include the subset of correct paraphrases.
(3) Using SPIKE
\cite{spike},\footnote{\url{https://spike.apps.allenai.org/}}
a search engine over Wikipedia sentences that supports
syntax-based queries, we searched for additional patterns
that appeared in Wikipedia and added them to 
\resource{}. Specifically, we searched for Wikipedia sentences
containing a  subject-object
tuple from T-REx and then manually extracted 
patterns from the  sentences. (4) Lastly, we added
additional paraphrases of the base-pattern
using the annotators' linguistic expertise. Two additional
experts went over all the patterns and corrected them, while
engaging in a discussion until reaching  agreement,
discarding patterns they could not agree on.

% (The annotators can also add entailing or entailed patterns, but we do not make use of them in this paper.)

% \nk{should we give a number?}.\ye{I don't think it's necessary}

% \nk{Not sure if the details of this need to be in the main paper or can be moved to the appendix?} \sr{I support moving the technicalities to an appendix and focus here on the properties of the resources and how it is being used.}
% \ye{can decide when we have the full story}
%The entailment edges were also annotated by the same two authors that created the patterns.




% \paragraph{Consistency Types}
% % \nk{we should change inference types to consistency types} \ye{but the graph is in entailment graph}
% To study the consistency against syntactic and lexical variations of models, each edge is augmented with the type of changes performed to reach from one pattern to another. We account for three types of changes: (1) \textit{syntactic}, (2) \textit{lexical} and (3) \textit{determiner}, all binary variables that account for a change of the specific type between two patterns.
% \textit{Syntactic} structure is defined as the dependency path between the subject and the object in a given pattern, where the path includes the edge types. Two patterns are considered equal syntactically if the labeled paths are identical.
% \textit{Lexical} difference is defined by words difference between two patterns, excluding determiners, punctuation, and symbols. The addition or removal of a preposition does not count as a lexical change.
% \textit{Determiner} difference is considered when the patterns' determiners do not fully overlap.



% and full statistics per relation are presented in Table \ref{tab:rel-graph-stats-elaborate} in the Appendix.


\paragraph{Human Agreement}
To assess the quality of \resource{}, we run a
human annotation study.
For each relation, we sample up to
five paraphrases, comparing each of the new patterns to the base-pattern that came with LAMA, which reflects the relation.
That is, given that relation $r_i$ contains the following patterns: $p_1, p_2, p_3, p_4$, and $p_1$ being the \textit{base-pattern}, we compare the following pairs $(p_1, p_2), (p_1, p_3), (p_1,p_4)$.

% By definition, two patterns that are paraphrases of the base pattern are also paraphrases of each other.

We populate the patterns with random subjects and objects from T-REx \cite{trex} and ask annotators if these sentences are paraphrases.
We also sample patterns from different relations to provide examples that are not paraphrases of each other, for control.
Each task contains five patterns that are thought
to be paraphrases, and two that are not.\footnote{The
  controls contain the same subjects and objects, so
  that only the pattern (not its argument) can be used to
  solve the task.}
Overall, we collect annotations for 156 paraphrase candidates and 61 control.

We asked five NLP graduate students to  annotate the pairs and collected one answer per pair.\footnote{Due to random annotation mistakes, for every question
that does not match our original label (either for
paraphrases or controls), we ask the annotators to relabel
them (without specifying the reason), to allow them to fix
random mistakes.}
% \footnote{Moreover, due to the sampling of patterns from different relations, seldom the patterns may be paraphrases of one another, which we do not count as mistakes.}
The agreement score for the paraphrases and the control are: 95.5\% and 98.3\%, which are high and reassures \resource's quality.
We further went through the disagreements  %of the paraphrases 
and fixed any additional problem %that we agree with, 
to further improve the resource quality.
% Examples of paraphrases are displayed in Table \ref{tab:predictions}. 
% \am{a bit odd to send the reader all the way to table 4, at this stage}
% \begin{table*}[t]
% \small
    \centering
\resizebox{1\textwidth}{!}{%
\begin{tabular}{llll}
\toprule
           Pattern & Base &  Entailed & Type \\
\midrule
% hi & byw & chao \\
      \textsc{Died-in} & \textsc{Subj} passed away in \textsc{Obj}. & $\Leftrightarrow$ \textsc{Subj} died in OBJ. & lexical \\
       & \textsc{Subj} was murdered at \textsc{Obj}. & $\Rightarrow$ \textsc{Subj} died in \textsc{Obj}. & lexical \& syntactic \\
       & \textsc{Subj} died at \textsc{Obj}. & $\Leftrightarrow$ \textsc{Subj} died in \textsc{Obj}. & lexical \\
       
       \midrule
       
       \textsc{Official-language} & \textsc{Obj} is the official language of \textsc{Subj}. & $\Leftrightarrow$ The official language of \textsc{Subj} is \textsc{Obj}. & syntactic \\
       & In \textsc{Subj} \textsc{Obj} is an official language. & $\Leftrightarrow$ The official language of \textsc{Subj} is \textsc{Obj}.	& \\
       
       \midrule
       
       \textsc{Located-in} & \textsc{Subj} is a county in \textsc{Obj}. & $\Rightarrow$ \textsc{Subj} is located in \textsc{Obj} . & lexical \& syntactic \\
       & \textsc{Subj} district, \textsc{Obj}.	& $\Rightarrow$ \textsc{Subj} is located in \textsc{Obj} . & lexical \& syntactic \\
       
       \midrule
       
       \textsc{Instrument} & \textsc{Subj} was a \textsc{Obj} player. & $\Rightarrow$ \textsc{Subj} plays \textsc{Obj}. & lexical \& syntactic \\
       & \textsc{Subj} played \textsc{Obj} . & $\Leftarrow$ \textsc{Subj} plays \textsc{Obj}. & tense \\
       & \textsc{Obj} sonatas of \textsc{Subj}. & $\Rightarrow$ \textsc{Subj} plays \textsc{Obj}. & lexical \& syntactic \\
       
\bottomrule
\end{tabular}
}
    \caption{Examples of patterns in the \resource{}.}
    \label{tab:rel-graph-examples}
\end{table*}

% One of the disagreements with our annotation arised from the following paraphrases: ``\textit{Royal Dutch Football Association} is a member of \textit{FIFA}.'' and ``\textit{Royal Dutch Football Association} is affiliated with the \textit{FIFA} organization.''
% Although these phrases are not paraphrases in the general case, we believe they are in our case with football association, and the other KB tuples from the KB. Such disagreement may arise from the lack of specific domain knowledge, but also from inherent disagreement in this sort of annotation that may arise from inherent disagreements \cite{pavlick2019inherent} or from different construals \cite{trott2020re}, which were recently observed in NLI \cite{elazar2020extraordinary}.
% \nk{I am not convinced that this paragraph is helping the paper. I think it might trigger people to doubt the resource}\yg{I agree. Either remove or prefix it by something that says this is an analysis of disagreements, and that there are only few of them.}

\begin{table}[t]
% \small
    \centering
\begin{tabular}{lr}
\toprule
           stat &   values \\
\midrule
      n. graphs &    15 \\
   avg patterns &    19.73 \\
   min patterns &     9 \\
   max patterns &    37 \\
       n. edges &  4821 \\
 edge syntactic &     0.03 \\
   edge lexical &     0.87 \\
      edge both &     0.09 \\
\bottomrule
\end{tabular}
    \caption{Statistics for \resource{}. \ye{numbers are not updates with all the graphs}}
    \label{tab:rel-graph-stats}
\end{table}